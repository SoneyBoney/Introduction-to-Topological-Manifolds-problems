%%%%%%%%%%%%%%%%%%%%%%%%%%%%%%%%%%%%%%%%%%%%%%%
%%%This is a science homework template. Modify the preamble to suit your needs. 
%The junk text is   there for you to immediately see how the headers/footers look at first 
%typesetting.


\documentclass[12pt]{article}

%AMS-TeX packages
\usepackage{amssymb,amsmath,amsthm} 
%geometry (sets margin) and other useful packages
\usepackage[margin=1.25in]{geometry}
\usepackage{tikz-cd}
\usepackage{graphicx,ctable,booktabs,mathrsfs}



%
%Redefining sections as problems
%
\makeatletter
\newenvironment{p}{\@startsection
       {section}
       {1}
       {-.2em}
       {-3.5ex plus -1ex minus -.2ex}
       {2.3ex plus .2ex}
       {\pagebreak[3]%forces pagebreak when space is small; use \eject for better results
       \large\bf\noindent{P }
       }
       }
\makeatother

\makeatletter
\newenvironment{s}{\@startsection
       {subsection}
       {2}
       {-.2em}
       {-3.5ex plus -1ex minus -.2ex}
       {2.3ex plus .2ex}
       {\pagebreak[3]%forces pagebreak when space is small; use \eject for better results
       \large\bf\noindent\emph{(sol) }
       }
       }
\makeatother

%
%Fancy-header package to modify header/page numbering 
%
\usepackage{fancyhdr}
\pagestyle{fancy}
%\addtolength{\headwidth}{\marginparsep} %these change header-rule width
%\addtolength{\headwidth}{\marginparwidth}
\chead{} 
\rhead{\thepage} 
\cfoot{} 
\renewcommand{\headrulewidth}{.3pt} 
\renewcommand{\footrulewidth}{.3pt}
\setlength\voffset{-0.25in}
\setlength\textheight{648pt}

%%%%%%%%%%%%%%%%%%%%%%%%%%%%%%%%%%%%%%%%%%%%%%%

%
%Contents of problem set
%    
\begin{document}

\title{Chapter 4 Exercises}
\author{Campinghedgehog}
\date{November 11, 2023}

\maketitle

\thispagestyle{empty}


\begin{p}{4.3}
    Suppose $X$ is a connected topological space, and $\sim$ is an equivalence relation on $X$ such that every
    equivalence class is open. Show that there is exactly one equivalence class, namely $X$ itself.
\end{p}
\begin{s}{4.3}
    By definition, equivalence relations partition a set into disjoint subsets. If there were more than 1 non-empty 
    partitions, then by proposition 4.1, $X$ would be disconnected.
    \qed
\end{s}

\begin{p}{4.4}
    Prove that a topological space $X$ is disconnected if and only if there exists a nonconstant continuous function
    from $X$ to the discrete space $\{0,1\}$.
\end{p}
\begin{s}{4.4}
    The contraposition of propostion 4.2 implies that if there exists a non-constant continuous function from $X$ 
    to $\{0,1\}$, then $X$ is disconnected. Thus we only need to show that if $X$ is disconnected, then there exists 
    a continuous non-constant function from $X\to \{0,1\}$.\\
    Let $X$ be disconnected. Then $\exists U,V \subseteq X$ such that $X = U \cup V$ and $U \cap V = \emptyset$.
    Then the function $f$ such that $f(U) = \{0\}$ and $f(V) = \{1\}$ is a continuous function because 
    $\{0\}$ is both open and closed and so is $U$. Same for $V$ and $\{1\}$. Furthermore $f$ is not constant.
    \qed
\end{s}

\begin{p}{4.5}
    Prove that a topological space is disconnected if and only if it is homeomorphic to a disjoint union of 
    two or more topological spaces.
\end{p}
\begin{s}{4.5}
    Let $X$ be a disconnected topological space. Then $\exists U,V \subseteq X$ such that $X = U \cup V$ and $U \cap V = \emptyset$.
    Let $Y = U \coprod V$. Then the identity map from $X \to Y$ where if $x\in U$ $x \mapsto (0,x)$
    and $y\in V, y\mapsto (1,y)$ is a homeomorphism between $X$ and $Y$. It is clearly bijective,
    then the open sets are also bijective since both $U$ and $V$ are open,
    $U$ and $V$ considered as topological spaces themselves have the same open sets as the subspace topology, which 
    has the same open sets as $X$ intersect either $U$ or $V$ respectively.\\
    Assume that $X$ is homeomorphic to a disjoint union of two or more topological spaces, call it $Y = U \coprod V$.
    Then $f^{-1}(U)$ is open and $f^{-1}(V)$ is open and they are disjoint and their union is all of $X$.
    \qed
\end{s}

\begin{p}{}
\end{p}
\begin{s}{}
\end{s}

\end{document}
