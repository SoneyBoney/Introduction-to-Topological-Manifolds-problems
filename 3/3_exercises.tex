%%%%%%%%%%%%%%%%%%%%%%%%%%%%%%%%%%%%%%%%%%%%%%%
%%%This is a science homework template. Modify the preamble to suit your needs. 
%The junk text is   there for you to immediately see how the headers/footers look at first 
%typesetting.


\documentclass[12pt]{article}

%AMS-TeX packages
\usepackage{amssymb,amsmath,amsthm} 
%geometry (sets margin) and other useful packages
\usepackage[margin=1.25in]{geometry}
\usepackage{graphicx,ctable,booktabs,mathrsfs}


%
%Redefining sections as problems
%
\makeatletter
\newenvironment{p}{\@startsection
       {section}
       {1}
       {-.2em}
       {-3.5ex plus -1ex minus -.2ex}
       {2.3ex plus .2ex}
       {\pagebreak[3]%forces pagebreak when space is small; use \eject for better results
       \large\bf\noindent{P }
       }
       }
\makeatother

\makeatletter
\newenvironment{s}{\@startsection
       {subsection}
       {2}
       {-.2em}
       {-3.5ex plus -1ex minus -.2ex}
       {2.3ex plus .2ex}
       {\pagebreak[3]%forces pagebreak when space is small; use \eject for better results
       \large\bf\noindent\emph{(sol) }
       }
       }
\makeatother

%
%Fancy-header package to modify header/page numbering 
%
\usepackage{fancyhdr}
\pagestyle{fancy}
%\addtolength{\headwidth}{\marginparsep} %these change header-rule width
%\addtolength{\headwidth}{\marginparwidth}
\chead{} 
\rhead{\thepage} 
\cfoot{} 
\renewcommand{\headrulewidth}{.3pt} 
\renewcommand{\footrulewidth}{.3pt}
\setlength\voffset{-0.25in}
\setlength\textheight{648pt}

%%%%%%%%%%%%%%%%%%%%%%%%%%%%%%%%%%%%%%%%%%%%%%%

%
%Contents of problem set
%    
\begin{document}

\title{Chapter 3 Exercises}
\author{Campinghedgehog}
\date{August 26, 2023}

\maketitle

\thispagestyle{empty}

\begin{p}{Exercise 3.12}
    (c) If $(p_i)$ is a sequence of points in $S$ and $p\in S$, then $p_i \to p$ in $S$ if and only if
    $p_i \to p$ in $X$.\\
    (d) Every subspace of a Hausdorff space is Hausdorff.\\
    (e) Every subspace of a first countable space is first countable.\\
    (f) Every subspace of a second countable space is second countable.
\end{p}
\begin{s}{Exercise 3.12}
    (c) Assume $p_i \to p$ in $S$. Since for all open neighborhoods $V_p$ in $S$, there exists $U_p$ in $X$ such
    that $V_p \subseteq U_p$, $p_i \to p$ in $X$ as well.\\
    Assume $p_i \to p$ in $X$. Since it is a sequence of points in $S$, by definition, for each open set intersect
    $S$, blah blah blah, $p_i \to p$ in $S$ as well.\\
    (d) Let $X$ be Hausdorff. Let $S$ be a subspace. Let $x,y \in S$. Then there exists
    $U_x$ an $U_y$ open sets of $X$ such that they are disjoint. Since they are disjoint, intersecting them 
    with $S$ still leaves them disjoint.\\
    (d) Every point has a neighborhood basis in $X$; taking the intersection of that neighborhood with $S$, 
    each basis element intersected with $S$ is open in $S$ and by definition is a basis for that neighborhood.\\
    (e) By part by, the basis for the subspace is the collection of basis elements intersected with $S$, which 
    is countable since the original basis is countable.
\end{s}


\begin{p}{Exercise 3.13}
       Let $X$ be a topological space and let $S$ be a subspace of $X$.
       Show that the inclusion map $S \hookrightarrow X$ is a topological embedding.
\end{p}
\begin{s}{Exercise 3.13}
       Restricting the co-domain to just $S$ yields the bijective identity, which is a homeomorphism.
\end{s}

\begin{p}{Exercise 3.14}
       Give an example of a topological embedding that is neither an open map nor a closed map.
\end{p}
\begin{s}{Exercise 3.14}
       The inclusion map is a topological embedding. So for any topological space $X$, choose a subset
       that is neither closed nor open, and thus the inclusion map from such a subset to $X$ is neither
       open nor closed.
\end{s}


\begin{p}{Exercise 3.15}
       A surjective topological embedding is a homeomorphism.
\end{p}
\begin{s}{Exercise 3.15}
       The restriction of a surjective topological embedding is the whole co-domain itself, thus the whole
       map is both injective and surjective so it's bijective and thus a homeomorphism with respect to
       the whole co-domain.
\end{s}




\begin{p}{Exercise 3.25}
    $$\mathscr{B}=\{U_1 \times \dots \times U_n : U_i \text{ is an open subset of } X_i,\, i=1,\dots ,n \}$$
    Prove that $\mathscr{B}$ is a basis for a topology.
\end{p}
\begin{s}{Exercise 3.25}
    Let $\mathscr{T}$ be the topology as follows: a subset of $X_1 \times \dots \times X_n$ is open if and 
    only if the projections into $X_1, \dots , X_n$ are open. \\
    By definition, each set in $\mathscr{B}$ is open. Let $U,V$ be arbitrary open subsets of the product space.
    Then it's components $U_i$s and $V_i$s are all open in $X_i$. Then $U_i \cap V_i$ is open in $X_i$.
    Then 
    $$U_1 \cap V_1 \times \dots \times U_n \cap V_n$$ 
    is open in the product space and is contained in 
    $$(U_1 \times ... \times U_n) \cap (V_1 \times ... \times V_n)$$
    Thus $\mathscr{B}$ is a basis for $\mathscr{T}$.
    \qed
\end{s}


\begin{p}{3.29}
    Prove the preceding corollary using only the characteristic property of the product topology.\\
    If $X_1,...,X_n$ are topological spaces, each canonical projection
    $\pi_i : X_1 \times \dots \times X_n \to X_i$ is continuous.
\end{p}
\begin{s}{3.29}
    Let $Y=X_1,...,X_n$ and let $f$ be the identity map. Since the identity is always continuous, 
    each $f_i = \pi_i \circ f$ is also continuous. And since $f$ is identity, $f_i = \pi_i$. So
    the result follows.
    \qed
\end{s}



\begin{p}{3.32}
    Prove proposition 3.31:\\
    Let $X_1,...,X_n$ be topological spaces.\\
    (a) The product topology is "associative" in the sense that the three topologies on the set  
    $X_1\times X_3 \times X_3$, obtained by thinking of it as 
    $X_1\times X_3 \times X_3$ or $(X_1\times X_3) \times X_3$ or $X_1\times (X_3 \times X_3)$
    are equal.\\
    (b) For any $i\in \{1,...,n\}$ and any points $x_j \in X_j$, $j \ne i$, the map 
    $X_i \to X_1 \times \dots \times X_n$ given by 
    $$f(x) = (x_1,\dots,x_{i-1},x,x_{i+1},\dots,x_n)$$
    is a topological embedding of $X_i$ into the product space.\\
    (c) Each canonical projection $\pi_i: X_1 \times \dots \times X_n \to X_i$ is an open map.\\
    (d) If for each $i$, $\mathscr{B}_i$ is a basis for the topology of $X_i$, then the set
    $$\{B_1 \times \dots \times B_n : B_i \in \mathscr{B}_i\}$$
    is abasis for the product topology.\\
    (e) If $S_i$ is a subspace of $X_i$ for $i=1,\dots ,n$, then the product topology and the subspace topology
    on $S_1 \times \dots \times S_n \subseteq X_1 \times \dots \times X_n$ are equal.\\
    (f) If each $X_i$ is Hausdorff, so is $X_1 \times \dots \times X_n$.\\
    (g) If each $X_i$ is first countable, so is $X_1 \times \dots \times X_n$.\\
    (h) If each $X_i$ is second countable, so is $X_1 \times \dots \times X_n$.
\end{p}
\begin{s}{3.32}
    (a) Let $f((x_1,x_2),x_3) = (x_1,x_2,x_3)$. Bijection is clear. Now just need to prove continuity 
    of $f$ and $f^{-1}$. Let $Y= (x_1,x_2),x_3)$ in the characteristic property. Then 
    $f_1 = \pi_1 \circ {\pi_1}_{(X_1 \times X_2)}$, $f_2 = \pi_1 \circ {\pi_2}_{(X_1 \times X_2)}$,
    $f_3 = id \circ  {\pi_2}_{(X_1 \times X_2)}$. Bascially a composition of projections.
    All $f_i$s are continuous because all canonical projects are are continuous and compositions of 
    continuous functions are continuous. Since all $f_i$ are continuous, $f$ is also continuous.
    The continuity of $f^{-1}$ can be proved similarly.\\
    (b) $f$ is continuous because for every open set in the product space, in particular $f^{-1}(U_i)$
    must be open by definition. $f^{-1}$ is continuous because it is the canonical projection.\\
    (c) Same reason as why $f$ is continuous as in (b)\\
    (d) (d) By definition the topology on the product space is the topology for which 
    $$\{U_1\times \dots \times U_n : U_i \text{ is open in } X_i\}$$
    is a basis. Since each $\mathscr{B}_i$ is a basis, each $U_i$ is a union of some family of 
    $B_i$s from $\mathscr{B}_i$. The result follows.\\
    (e) Open-ness is defined on the product space by conjuction (ands) and subspace topology is the intersection
    of open sets with a subset -- and since intersection boils down to conjuction of set elements -- and since 
    conjuction is associative and commutative, the result follows. In order words, conjuctions of intersections 
    is equal to intersections of conjunctions (of open sets).\\
    (f) Let $(x_1,\dots,x_n)$ and $(y_1,\dots,y_n)$ be two distinct points of the product space. By assumption,
    for all $i$, there exists $U_i$ and $V_i$ open neighborhoods of $x_i$ and $y_i$ respectively, such that
    $U_i \cap V_i$ is empty. Since all $U_i$ and $V_i$ are open, $U_i \times \dots \times U_n$ and 
    $V_i \times \dots \times V_n$  are open subsets of the product space. Moreover they are disjoint since each 
    component in the cartesian product is disjoint.\\
    (g) If each point in $X_i$ has a countable neighborhood basis, the cartesian product is also a neighborhood
    basis by definition of product topology, but it must also be countable because a k-cell of countable sets must
    also be countable from elementary set theory.\\
    (h) Countability is preserved for same reason as in (g)
\end{s}


\begin{p}{3.34}
    Suppose $f_1,f_2: X \to \mathbb{R}$ are continuous functions. Their pointwise sum 
    $f_1 + f_2: X \to \mathbb{R}$ and pointwise product $f_1f_2: X \to \mathbb{R}$ are real-valued functions 
    defined by
    $$(f_1+f_2)(x) = f_1(x) + f_2(x), \qquad (f_1f_2)(x)=f_1(x)f_2(x)$$
    Use the characteristic property of the product topology to show that pointwise sum and products of continuous
    functions are continuous.
\end{p}
\begin{s}{3.34}
    Using the characteristic property, since $f_1,f_2$ are both continuous, then the mapping 
    $f: X \to \mathbb{R} \times \mathbb{R}$ is also continuous. Since multiplication and addition are both 
    continuous, i.e. $m,a: \mathbb{R} \times \mathbb{R} \to \mathbb{R}$, the composition of $f$ and mult 
    or add is also continuous.
    \qed
\end{s}


\begin{p}{3.38}
    (Characteristic property of Infinite product spaces)\\
    Let $(X_\alpha)_{\alpha\in A}$ be an indexed family of topological spaces. For any topological space $Y$,
    a map $f: Y \to \prod_{\alpha\in A}X_\alpha$ is continuous if and only if each of its component functions
    $f_\alpha=\pi_\alpha \circ f$ is continuous. The product topology is the unique topology on
    $\prod_{\alpha\in A}X_\alpha$  that satisfies this property.
\end{p}
\begin{s}{3.38}
    The product topology is the only topology to satisfy this property because of the finite intersection
    of open sets are open property of topologies. The preimages of infinite spaces in the product space 
\end{s}


\begin{p}{3.40}
    Show that the disjoint union topology is indeed a topology.
\end{p}
\begin{s}{3.40}
    The whole disjoint space is open since the intersection with each $X_\alpha$ is equal to $X_\alpha$ which
    is open since $X_\alpha$ is a topological space. As is the empty set for the same reason. 
    Let $U,V$ be open sets in the disjoint union space. Then $U \cap V$ is open since 
    $(X_\alpha \cap U) \cap (X_\alpha \cap V) = X_\alpha \cap (U \cap V)$. Openness of finite intersections
    follows from induction. An arbitrary union of open sets in the product space is also open since
    The $X_\alpha$ are all disjoint so each $X_\alpha \cap U$ will be open and an arbitrary union of those 
    is open.
    \qed
\end{s}


\begin{p}{3.43}
    Let $(X_\alpha)_{\alpha\in A}$ be an indexed family of topological spaces.\\
    (a) A subset of $\coprod_{\alpha \in A}X_\alpha$ is closed if and only if its intersection with each
    $X_\alpha$ is closed.\\
    (b) Each canonical injection $\iota_\alpha : X_\alpha \to \coprod_{\alpha \in A}X_\alpha$ is
    a topological embedding and an open and closed map.\\
    (c) If each $X_\alpha$ is Hausdorff, then so is $\coprod_{\alpha \in A}X_\alpha$.\\
    (d) If each $X_\alpha$ is first countable, then so is $\coprod_{\alpha \in A}X_\alpha$.\\
    (e) If each $X_\alpha$ is second countable and the index set $A$ is countable, then 
    $\coprod_{\alpha \in A}X_\alpha$ is second countable.
\end{p}
\begin{s}{3.43}
    (a) Assume that a subset $C$ in the disjoint union space is closed. Then 
    $\coprod_{\alpha \in A}X_\alpha \setminus C$ is open which mean that
    $X_\alpha \cap (\coprod_{\alpha \in A}X_\alpha \setminus C)$ is open in $X_\alpha$.
    Which is equal to $X_\alpha \cap (\coprod_{\alpha \in A}X_\alpha \cap C^\mathsf{c})$
    = 
    $(X_\alpha \setminus C) \cap \coprod_{\alpha \in A}X_\alpha$ being open 
    in $X_\alpha$. Which implies $C$ is closed in $X_\alpha$.\\
    Let $C$ be such that each intersection with each $X_\alpha$ is closed.
    Then $X_\alpha \setminus C$ is open. And the result follows similarly as above but in reverse.\\
    (b) Since the second element of the pair is all the same if we restrict the  co-domain to the image,
    we bascially get the identity map which is a homeomorphism and is both open and closed.\\
    (c) Let each $X_\alpha$ be Hausdorff. Let $x,y$ be distinct points in the disjoint union space.
    All $X_\alpha$s are disjoint by definition so there are just two cases:\\
    1) $x,y$ are in the same $X_\alpha$. Then since $X_\alpha$ is Hausdorff, there exists disjoint open 
    neighborhoods of $x$ and $y$ respectively.\\
    2) If $x,y$ are not in the same $X_\alpha$, then their respective
    $X_\alpha$ is the open neighborhoods that are by definition disjoint.\\
    (d) Let $x$ be any point of the disjoint union space. Then by definition it must be a point in exactly
    one $X_\alpha$. Since $X_\alpha$ is first countable, there exists a countable neighborhood basis of 
    $x$ in $X_\alpha$. By definition of disjoint union topology, that neighborhood basis in $X_\alpha$ 
    is also a neighborhood basis in the disjoint union topology/space, so the disjoint union space is 
    also first countable.\\
    (e) This follows since if $A$ is countable then we get a countable union of countable sets which is 
    itself countable.
    \qed
\end{s}


\begin{p}{3.44}
    Suppose $(X_\alpha)_{\alpha\in A}X_\alpha$ is an indexed family of nonempty $n$-manifolds. 
    Show that the disjoint union $\coprod_{\alpha \in A}X_\alpha$ is an $n$-manifold if and only if 
    $A$ is countable. 
\end{p}
\begin{s}{3.44}
    Hausdorffness and second-Countability follow from Proposition 3.42. Only need to show local euclidean. 
    For each $x$, it will fall into exactly one $X_\alpha$ which is locally euclidian and thus so is 
    the disjoint union space since the inclusion map is a topological embedding and is open and closed 
    mapping.
    \qed
\end{s}

\begin{p}{3.45}
    Let $X$ be any space and $Y$ be a discrete space. Show that the Cartesian product $X\times Y$ is equal
    to the disjoint union $\coprod_{y \in Y} X_y$, and the product topology is the sameas the disjoint union
    topology.
\end{p}
\begin{s}{3.45}
    It is clear that they have the same elements.\\
    Since $Y$ is a discrete space, every subset of $Y$ is open. In particular the set of singletons in $Y$
    form a basis for $Y$. Then $\{U, {y}\}$ where $U$ is open in $X$ and $y\in Y$ forms a basis for 
    the product space. But then in the disjoint union space a set is open if the restriction is open, which
    boils down to $|Y|$ copies of $X$s indexed by $y$, which is exactly the basis above. Thus they have
    the same topologies.
    \qed
\end{s}

\begin{p}{3.46}
    Show that the quotient topology is indeed a topology.
\end{p}
\begin{s}{3.46}
    Empty set is open trivialy. The whole $Y$ is open by surjection. And the fact that
    intersection and union pass through preimages imply that the quotient topology is a topology.
    \qed
\end{s}


\begin{p}{3.55}
    Show that every wedge sum of Hausdorff spaces is Hausdorff.
\end{p}
\begin{s}{3.55}
    There are two cases. The case where neither distinct point is the collapsed base point, and the case where 
    one is.\\
    Case 1) If neither distinct points is the collapsed point, then Hausdorffness follows from Hausdorffness 
    of disjoint union space.\\
    Case 2) Let $x$ be the collapsed base point and $y$ be any other point. The preimage of $x$ will then be
    $\{p_\alpha\}_{\alpha\in A}$. Then for each of these $p_\alpha$, since they come from Hausdorff spaces,
    there exists an open neighborhood that is disjoint from the open neighborhood of $y$. Then the union of 
    all the open neighborhoods of all the $p_\alpha$ is itself open, which means that image of the quotient 
    map of such a set is also open in the quotient space and is disjoint from the open neighborhood of $y$.
    \qed
\end{s}

\begin{p}{3.59}
    Let $q: X \to Y$ be any map. For a buset $U\subseteq X$, show that the following are equivalent.\\
    (a) $U$ is saturated.\\
    (b) $U = q^{-1}(q(U))$.\\
    (c) $U$ is an union of fibers.\\
    (d) If $x\in U$, then every point $x^\prime\in X$ such that $q(x)=q(x^\prime)$ is also in $U$.
\end{p}
\begin{s}{3.59}
    $(a \implies b)$ Assume $U$ is saturated. Then there exists some $V \subseteq Y$ such that 
    $q^{-1}(Y) = U$. This assumption rules out the case where the image of $U$ contains a point where more than 
    one element of $X$ maps to, where one of those points is not in $U$ (where $q$ is not injective). The result
    follows.\\
    $(b \implies c)$ Assume $U = q^{-1}(q(U))$. This can be rewritten as 
    $$U = \bigcup \{q^{-1}(y) : y \in q(U)\}$$
    $(c \implies d)$ Assume $U$ is a union of fibers. Let $x \in U$. Let $x^\prime \in X$ be such that 
    $q(x) = q(x^\prime)$. Since $U$ is a union of fibers, and since $q(x) = q(x^\prime)$, they map to the same
    element in the image, which means they are in $U$.
    $(d \implies a)$ Since by definition $U$ contains each fiber, there exists some subset in the co-domain that
    has the pre-image as $U$.
    \qed
\end{s}


\begin{p}{3.61}
    A continuous surjective map $q: X \to Y$ is a quotient map if and only if it takes saturated open subsets to
    open subsets, or saturated closed subsets to closed subsets.
\end{p}
\begin{s}{3.61}
    If $q$ is a quotient map, then the target's open sets are the ones for which its preimage is open in the domain.
    Let $U$ be a saturated open set in $X$. Then we have that $U=q^{-1}(q(U))$. Then $q(U)$ is open since $q$ is a 
    quotient map. Therefore, $q$ takes saturated open sets to open sets.\\
    If $q$ takes saturated open sets to open sets. Let $V \subseteq Y$. Need to show that $V$ is open if and only 
    if $q^{-1}(V)$ is open. If $V$ is open, then $q^{-1}(V)$ is open by continuity of $q$. Assume that 
    $q^{-1}(V)$ is open. By definition $q^{-1}(V)$ is a union of fibers, which means it is a saturated open set,
    so $V$ is open.
    \qed
\end{s}



\begin{p}{3.63}
    (a) Any composition of quotient maps is a quotient map.\\
    (b) An injective quotient map is a homeopmorphism.\\
    (c) If $q: X \to Y$ is a quotient map, a subset $K \subseteq Y$ is closed if and only if $q^{-1}(K)$
    is closed in $X$.\\
    (d) If $q: X \to Y$ is a quotient map and $U \subseteq X$ is a saturated open or closed subset, then 
    the restriction $\left.q\right|_U: U \to q(U)$ is a quotient map.\\
    (e) If $\{q_\alpha : X_\alpha \to Y_\alpha\}_{\alpha \in A}$ is an indexed family of quotient maps, then 
    the map $q: \coprod_\alpha X_\alpha \to \coprod_\alpha Y_\alpha$ whose restriction to each $X_\alpha$ 
    is equal to $q_\alpha$ is a quotient map.
\end{p}
\begin{s}{3.63}
    (a) Let $f: X \to Y$, $g: Y \to Z$ be quotient maps. Then the composition is surjective as well.
    Composition of continuity shows that the $V$ open in $Z$ implies $(f\circ g)^{-1}(V)$ open in $X$.
    Let $U$ be a saturated open set in $X$. Then $U= f^{-1}(g^{-1}(V))$ for some $V \subseteq Z$. Need to
    show that $V$ is open. Then $f(U) = g^{-1}(V)$ and $f(U)$ is open and also a union of fibers, Therefore
    $g(f(U)) = V$ is open in $Z$.\\
    (b) A quotient map is already surjective. So a map that is both injective and surjective is bijective 
    by the Cantor-Bernstein theorem. By definition it is continuous. By injective, all subsets are 
    saturated, thus it is an open map, so it is a homeopmorphism.\\
    (c) Let $K\subseteq Y$ be closed. Then $q^{-1}(K)$ is closed by continuity. 
    Let $q^{-1}(K)$  be closed in $X$. Then by definition it is a saturated subset, and so it's image, $K$
    is closed.\\
    (d) The restriction is surjective by definition. It is also continuous by the local criterion for continuity.
    Because $U$ is a saturated set, any subset of it is also saturated. Thus open subset of $U$ must also map
    to an open set of $q(U)$, so $q$ is a quotient map.\\
    (e) Let $V \subseteq \coprod_\alpha Y_\alpha$ be open. Then the intersection with each $Y_\alpha$ is open.
    Then $q_\alpha^{-1}(V_\alpha)$ is open for all $\alpha$, thus $q^{-1}(V_\alpha)$ is open. The other way is similar
    following from the fact that a saturated open subset's intersection with each subspace is also saturated.
    \qed
\end{s}


\begin{p}{3.72}
    (Uniqueness of the Quotient Topology)\\
    Given a topological space $X$, a set $Y$, and a surjective map $q: X \to Y$, the quotient topology is the only 
    topology on $Y$ for which the characteristic property holds.
\end{p}
\begin{s}{3.72}
    Let $\tau_q,\tau_p$ be two topologies for which the property holds, where $\tau_q$ is the quotient topology.
    Let $Z=Y_p$ and $f_q: Y_q \to Y_p$ be the identity map. Simlarly let $Z=Y_q$ and $f_p: Y_p \to Y_q$ be the 
    identity. Then if both are continuous then the result follows. \\
    $f_p \circ q_p: X \to Y_q$ is continuous (quotient map), and thus $f_p$ is continuous. \\
    Let $q': X \to Y_p$, and $id: Y_d \to Y_d$. $id$ is clearly continuous. Thus using the characteristic property,
    $q' = id \circ q'$ is continuous. But also $q' = f_p \circ q$, using the characteristic property again,
    $f_p$ is continuous.
    \qed
\end{s}


\begin{p}{3.83}
    Verfiy each of the following to be a topological group:\\
    (a) the real line $\mathbb{R}$ with its additive group structure and Euclidean topology \\
    (b) the set $\mathbb{R} \setminus \{0\}$ of nonzero real numbers under multiplication, with the Euclidean topology\\
    (c) the general linear group $GL(n,\mathbb{R})$, which is the set of $n \times n$ invertible real matrices
    under matrix multiplication, with the subspace topology obtained from $\mathbb{R}^{n^2}$ (where we identify 
    an $n\times n$ matrix with a point in $\mathbb{R}^{n^2}$ by using the matrix entries as coordinates)\\
    (d) any group whatsoever with the discrete topology (any such group is called a discrete group)
\end{p}
\begin{s}{3.83}
    (a) Addition is continuous by limit laws, negation is continuous as well. \\
    (b) Same as part a\\
    (c) idk\\
    (d) by assumption it's a group. Continuity follows since any function from a discrete topology will be 
    continuous.
\end{s}


\begin{p}{3.85}
    Show that any subgroup of a topological group is a topological group with the subspace topology.
    Any finite product of topological groups is a topological group with the direct product group
    structure and the product topology.
\end{p}
\begin{s}{3.85}
    Let $G$ be a topological group. Let $S \leq G$ be a subgroup. Then by definition $S$ is a group. Just need 
    to show that negation and group operation are continuous. Since $S$ is a subset it inherits the subset topology.
    For any open subset in the image $V \subseteq G$, $V \cap S$ is open in $S$. Then $m^{-1}(V) \cap S$ is open,
    same as negation, so $S$ is a topological group.\\
    Let $G_1, \dots , G_n$ be topological groups. Then $G_1 \times \dots \times G_n$ is a group with the 
    direct product group operation. Also imbue the space in the product topology. Since $M$ is the group operation,
    it can be written as $(m_1,\dots,m_n)(x)=(m_1(x),\dots,m_n(x))$, where each $m_i$ is continuous. By proposition
    3.33, $M$ is continuous; same goes for negation. Thus the product space is a topological group.
    \qed
\end{s}


\begin{p}{}
\end{p}
\begin{s}{}
\end{s}



\begin{p}{}
\end{p}
\begin{s}{}
\end{s}

\end{document}
