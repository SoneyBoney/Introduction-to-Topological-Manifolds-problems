%%%%%%%%%%%%%%%%%%%%%%%%%%%%%%%%%%%%%%%%%%%%%%%
%%%This is a science homework template. Modify the preamble to suit your needs. 
%The junk text is   there for you to immediately see how the headers/footers look at first 
%typesetting.


\documentclass[12pt]{article}

%AMS-TeX packages
\usepackage{amssymb,amsmath,amsthm} 
%geometry (sets margin) and other useful packages
\usepackage[margin=1.25in]{geometry}
\usepackage{graphicx,ctable,booktabs,mathrsfs}


%
%Redefining sections as problems
%
\makeatletter
\newenvironment{p}{\@startsection
       {section}
       {1}
       {-.2em}
       {-3.5ex plus -1ex minus -.2ex}
       {2.3ex plus .2ex}
       {\pagebreak[3]%forces pagebreak when space is small; use \eject for better results
       \large\bf\noindent{P }
       }
       }
\makeatother

\makeatletter
\newenvironment{s}{\@startsection
       {subsection}
       {2}
       {-.2em}
       {-3.5ex plus -1ex minus -.2ex}
       {2.3ex plus .2ex}
       {\pagebreak[3]%forces pagebreak when space is small; use \eject for better results
       \large\bf\noindent\emph{(sol) }
       }
       }
\makeatother

%
%Fancy-header package to modify header/page numbering 
%
\usepackage{fancyhdr}
\pagestyle{fancy}
%\addtolength{\headwidth}{\marginparsep} %these change header-rule width
%\addtolength{\headwidth}{\marginparwidth}
\chead{} 
\rhead{\thepage} 
\cfoot{} 
\renewcommand{\headrulewidth}{.3pt} 
\renewcommand{\footrulewidth}{.3pt}
\setlength\voffset{-0.25in}
\setlength\textheight{648pt}

%%%%%%%%%%%%%%%%%%%%%%%%%%%%%%%%%%%%%%%%%%%%%%%

%
%Contents of problem set
%    
\begin{document}

\title{Title}
\author{Author}
\date{Date}

\maketitle

\thispagestyle{empty}

\begin{p}{3.1}
    Suppose $M$ is an $n$-dimensional manifold with boundary. 
    Show that $\partial M$ is an $(n-1)$-manifold (without boundary) when endowed with the subspace topology.
    You may use without proof the fact that $Int$ M and $\partial M$ are disjoint.
\end{p}
\begin{s}{3.1}
    Hausdorffness and second countability from the subspace topology is inherited from the topology on $M$.
    To prove that the boundary of $M$ is locally euclidean of dimension $(n-1)$, we use the assumption that 
    $M$ is a manifold with boundary, this means that for all points $p\in M$, there exists an open neighborhood
    $U$ that is homeomorphic to $\mathbb{H}^n$. If $p$ is in the boundary, then given that the interior and boundary
    of $M$ are disjoint, we have that $U \cap \partial M$ is homeomorphic to $\mathbb{R}^(n-1) \times {0}$, which
    admits an embedding from $\mathbb{R}^(n-1)$. Thus an point on the boundary of $M$ is locally euclidean of 
    dimension $n-1$.
\end{s}

\begin{p}{3.2}
    Suppose $X$ is a topological space and $A\subseteq B \subseteq X$ Show that $A$ is dense in $X$ if and only if
    $A$ is dense in $B$ and $B$ is dense in $X$.
\end{p}
\begin{s}{3.2}
    Assume that $A$ is dense in $X$, i.e. $\overline A = X$. From exercise 3.7 we have that the closure of $A$ in
    $B$ is equal to the closure of $A$ in $X$ intersected with $B$, and since $\overline A = X$, the closure of $A$
    in $B$ is equal to $B$. The closure of $B$ in $X$ contains the closure of $A$, i.e. 
    $\overline A = X \subseteq \overline B$, since $B$ is a subset we have that $\overline B = X$.\\
    Conversely, assume that $A$ is dense in $B$ and $B$ is dense in $X$. $\overline B$ is the smallest closed
    set that contains $B$, any closed set that contains $B$ must contain $\overline B$. By assumption we have
    that the closure of $A$ in $B$ is equal to $B$. So $\overline A$ in $X$ is a closed set in $X$ that contains $B$.
    Thus $\overline A$ will contain $\overline B = X$. Thus $A$ is dense in $X$.
    \qed
\end{s}

\begin{p}{3.3}
    Show by giving a counterexample that the conclusion of the gluing lemma (Lemma 3.23) need not hold if 
    $\{A_i\}$ is an infinite closed cover.
\end{p}
\begin{s}{3.3}
    Let $A_i = [\frac{1}{i}, i]$, then the last part of the proof of the gluing lemma does not hold because taking
    the union of these, and using de morgans laws, we see that the unions of $A_i$ for $i \in \mathbb{N}$
    is not closed.
    \qed
\end{s}


\begin{p}{3.4}
    Show that every closed ball in $\mathbb{R}^n$ is an $n$-dimensional manifold with boundary, as the complement
    of every open ball. Assuming the theorem on the invariance of the boundary, show that the manifold boundary of 
    each is equal to its topological boundary as a subset of $\mathbb{R}^n$, namely a sphere.
    [Hint: for the unit ball in 
    $\mathbb{R}^n$, consider the map $\pi \circ \sigma^{-1}: \mathbb{R}^n \to \mathbb{R}^n$], where $\sigma$
    is the stereographic projection and $\pi$ is a projection from $\mathbf{R}^{n+1}$ to $\mathbb{R}^n$
    that omits some coordinate other than the last.
\end{p}
\begin{s}{3.4}
    Let $B$ be a closed ball in $\mathbb{R}^n$. For points in the interior of $B$, the inclusion map into 
    $\mathbb{R}^n$ is a chart. For points on the boundary, the idea is to use the inverse of the stereographic
    projection to lift the boundary onto the surface, then project it back down by throwing away the last 
    Skip for now.
\end{s}

\begin{p}{3.5}
    Show that a finite product of open maps is open;
    give a counterexample to show that a finite product of closed maps need not be closed.
\end{p}
\begin{s}{3.5}
    The product of open maps is open, and also each $f(U)$ is open, so the product of them are also open.
    Two copies of $f(x)=max(x,0)$ is a counterexample. The mapping is closed because it's a union of 
    the preimage and $\{0\}$. But the product of two of these maps, the set $\{xy | xy = -1\}$ has an image that 
    is not closed.
    \qed
\end{s}

\begin{p}{3.6}
    Let $X$ be a topological space. The diagonal of $X \times X$ is the subset
    $\Delta = \{(x,x): x \in X\} \subseteq X \times X$. Show that $X$ is Hausdorff if and only if 
    $\Delta$ is closed in $X \times X$.
\end{p}
\begin{s}{3.6}
    Assume that $X$ is Hausdorff. We want to show that $\Delta^\complement = (X \times X) \setminus \Delta$ is open.
    We will do this by showing that every point in $\Delta^\complement$ has an open neighborhood contained in
    $\Delta^\complement$. Let $(p,q) \in \Delta^\complement$, then $p \ne q$. Since $X$ is Hausdorff, there 
    exists respective open neighborhoods $U,V$ of $p,q$ such that $U \cap V = \emptyset$. Since both $U,V$ are open
    their cartesian product is open in the product space, and is contained in $\Delta^\complement$ since they are 
    disjoint. Therefore $\Delta$ is closed. \\
    The other direction is the exact reasoning just reversed.
    \qed
\end{s}

\begin{p}{3.7}
    Show that the space $X$ of Problem $2-22$ is homeomorphic to $\mathbb{R}_d \times \mathbb{R}$ where 
    $\mathbb{R}_d$ is the set $\mathbb{R}$ with the discrete topology.
\end{p}
\begin{s}{3.7}
    We can show that the component spaces are homeomorphic. By definition the first component of the space 
    in problem 2-22 the element can be anything so it has the discrete topology. The second component is
    $a \le y \le b$ which is the same as $x < b-a = \epsilon$ for some $\epsilon >0$ which is the same as
    the euclidean topology.
    \qed
\end{s}

\begin{p}{3.8}
    Let $X$ denote the Cartesian product of countably infinite many copies of $\mathbb{R}$
    (which is just the set of all infinite sequence of real numbers), endowed with the box topology. 
    Define a map $f: \mathbb{R} \to X$ by $f(x)= (x,x,x,\dots)$.
    Show that $f$ is not continuous, even though each of its component functions is.
\end{p}
\begin{s}{3.8}
    Consider the cardinality of the bases of the domain and co-domain. We know $\mathbb{R}$ is second countable so
    it has a countable basis. On the other hand, the infinite sequence of countable sets is not countable (which
    can be shown by diagonalization argument). Since the codomain has strictly more open sets than the domain, 
    the function cannot be continuous.
    \qed
\end{s}

\begin{p}{3.9}
    Let $X$ be as in the preceding problem. Let $X^+ \subseteq X$ be the subset consisting of the sequences 
    of strictly positive real numbers, and let $z$ denote the zero sequence, that is, the one whose terms are 
    $z_i = 0$ for all $i$. Show taht $z$ is in the closure of $X^+$, but there is no sequence of elements of
    $X^+$ converging to $z$. Then use the sequence lemma to conclude that $X$ is not first countable, and thus
    not metrizable.
\end{p}
\begin{s}{3.9}
    TODO
\end{s}

\begin{p}{problem 3.10}
    Suppose that $(X_\alpha)_{\alpha \in A}$ is an indexed family of topological spaces, and $Y$ is an topological
    space. A map $f: \coprod_{\alpha \in A}X_\alpha \to Y$ is continuous if and only if its restriction to each 
    $X_\alpha$ is continuous. The disjoint union topology is the unique topology on $\coprod_{\alpha \in A}X_\alpha$ 
    with this property.
\end{p}
\begin{s}{problem 3.10}
    The gluing lemma shows that if each restriction is continuous, then $f$ is continuous.\\
    Assume that $f$ is continuous. Then the restriction to each $X_\alpha$ must also be continuous because
    each open set in $X_\alpha$ is also open in the disjoint union space since $X_\alpha$ is by definition open
    so any open set intersected with it is also open.\\
    The disjoint union topology is the unique topology with this property because...
\end{s}


\begin{p}{3.11}
    
\end{p}
\begin{s}{3.11}
\end{s}


\begin{p}{}
\end{p}
\begin{s}{}
\end{s}

\end{document}
