%%%%%%%%%%%%%%%%%%%%%%%%%%%%%%%%%%%%%%%%%%%%%%%
%%%This is a science homework template. Modify the preamble to suit your needs. 
%The junk text is   there for you to immediately see how the headers/footers look at first 
%typesetting.


\documentclass[12pt]{article}

%AMS-TeX packages
\usepackage{amssymb,amsmath,amsthm} 
%geometry (sets margin) and other useful packages
\usepackage[margin=1.25in]{geometry}
\usepackage{tikz-cd}
\usepackage{graphicx,ctable,booktabs,mathrsfs}


%
%Redefining sections as problems
%
\makeatletter
\newenvironment{p}{\@startsection
       {section}
       {1}
       {-.2em}
       {-3.5ex plus -1ex minus -.2ex}
       {2.3ex plus .2ex}
       {\pagebreak[3]%forces pagebreak when space is small; use \eject for better results
       \large\bf\noindent{P }
       }
       }
\makeatother

\makeatletter
\newenvironment{s}{\@startsection
       {subsection}
       {2}
       {-.2em}
       {-3.5ex plus -1ex minus -.2ex}
       {2.3ex plus .2ex}
       {\pagebreak[3]%forces pagebreak when space is small; use \eject for better results
       \large\bf\noindent\emph{(sol) }
       }
       }
\makeatother

%
%Fancy-header package to modify header/page numbering 
%
\usepackage{fancyhdr}
\pagestyle{fancy}
%\addtolength{\headwidth}{\marginparsep} %these change header-rule width
%\addtolength{\headwidth}{\marginparwidth}
\chead{} 
\rhead{\thepage} 
\cfoot{} 
\renewcommand{\headrulewidth}{.3pt} 
\renewcommand{\footrulewidth}{.3pt}
\setlength\voffset{-0.25in}
\setlength\textheight{648pt}

%%%%%%%%%%%%%%%%%%%%%%%%%%%%%%%%%%%%%%%%%%%%%%%

%
%Contents of problem set
%    
\begin{document}

\title{Title}
\author{Author}
\date{Date}

\maketitle

\thispagestyle{empty}

\begin{p}{3.1}
    Suppose $M$ is an $n$-dimensional manifold with boundary. 
    Show that $\partial M$ is an $(n-1)$-manifold (without boundary) when endowed with the subspace topology.
    You may use without proof the fact that $Int$ M and $\partial M$ are disjoint.
\end{p}
\begin{s}{3.1}
    Hausdorffness and second countability from the subspace topology is inherited from the topology on $M$.
    To prove that the boundary of $M$ is locally euclidean of dimension $(n-1)$, we use the assumption that 
    $M$ is a manifold with boundary, this means that for all points $p\in M$, there exists an open neighborhood
    $U$ that is homeomorphic to $\mathbb{H}^n$. If $p$ is in the boundary, then given that the interior and boundary
    of $M$ are disjoint, we have that $U \cap \partial M$ is homeomorphic to $\mathbb{R}^(n-1) \times {0}$, which
    admits an embedding from $\mathbb{R}^(n-1)$. Thus an point on the boundary of $M$ is locally euclidean of 
    dimension $n-1$.
\end{s}

\begin{p}{3.2}
    Suppose $X$ is a topological space and $A\subseteq B \subseteq X$ Show that $A$ is dense in $X$ if and only if
    $A$ is dense in $B$ and $B$ is dense in $X$.
\end{p}
\begin{s}{3.2}
    Assume that $A$ is dense in $X$, i.e. $\overline A = X$. From exercise 3.7 we have that the closure of $A$ in
    $B$ is equal to the closure of $A$ in $X$ intersected with $B$, and since $\overline A = X$, the closure of $A$
    in $B$ is equal to $B$. The closure of $B$ in $X$ contains the closure of $A$, i.e. 
    $\overline A = X \subseteq \overline B$, since $B$ is a subset we have that $\overline B = X$.\\
    Conversely, assume that $A$ is dense in $B$ and $B$ is dense in $X$. $\overline B$ is the smallest closed
    set that contains $B$, any closed set that contains $B$ must contain $\overline B$. By assumption we have
    that the closure of $A$ in $B$ is equal to $B$. So $\overline A$ in $X$ is a closed set in $X$ that contains $B$.
    Thus $\overline A$ will contain $\overline B = X$. Thus $A$ is dense in $X$.
    \qed
\end{s}

\begin{p}{3.3}
    Show by giving a counterexample that the conclusion of the gluing lemma (Lemma 3.23) need not hold if 
    $\{A_i\}$ is an infinite closed cover.
\end{p}
\begin{s}{3.3}
    Let $A_i = [\frac{1}{i}, i]$, then the last part of the proof of the gluing lemma does not hold because taking
    the union of these, and using de morgans laws, we see that the unions of $A_i$ for $i \in \mathbb{N}$
    is not closed.
    \qed
\end{s}


\begin{p}{3.4}
    Show that every closed ball in $\mathbb{R}^n$ is an $n$-dimensional manifold with boundary, as the complement
    of every open ball. Assuming the theorem on the invariance of the boundary, show that the manifold boundary of 
    each is equal to its topological boundary as a subset of $\mathbb{R}^n$, namely a sphere.
    [Hint: for the unit ball in 
    $\mathbb{R}^n$, consider the map $\pi \circ \sigma^{-1}: \mathbb{R}^n \to \mathbb{R}^n$], where $\sigma$
    is the stereographic projection and $\pi$ is a projection from $\mathbf{R}^{n+1}$ to $\mathbb{R}^n$
    that omits some coordinate other than the last.
\end{p}
\begin{s}{3.4}
    Let $B$ be a closed ball in $\mathbb{R}^n$. For points in the interior of $B$, the inclusion map into 
    $\mathbb{R}^n$ is a chart. For points on the boundary, the idea is to use the inverse of the stereographic
    projection to lift the boundary onto the surface, then project it back down by throwing away the last 
    Skip for now.
\end{s}

\begin{p}{3.5}
    Show that a finite product of open maps is open;
    give a counterexample to show that a finite product of closed maps need not be closed.
\end{p}
\begin{s}{3.5}
    The product of open maps is open, and also each $f(U)$ is open, so the product of them are also open.
    Two copies of $f(x)=max(x,0)$ is a counterexample. The mapping is closed because it's a union of 
    the preimage and $\{0\}$. But the product of two of these maps, the set $\{xy | xy = -1\}$ has an image that 
    is not closed.
    \qed
\end{s}

\begin{p}{3.6}
    Let $X$ be a topological space. The diagonal of $X \times X$ is the subset
    $\Delta = \{(x,x): x \in X\} \subseteq X \times X$. Show that $X$ is Hausdorff if and only if 
    $\Delta$ is closed in $X \times X$.
\end{p}
\begin{s}{3.6}
    Assume that $X$ is Hausdorff. We want to show that $\Delta^\complement = (X \times X) \setminus \Delta$ is open.
    We will do this by showing that every point in $\Delta^\complement$ has an open neighborhood contained in
    $\Delta^\complement$. Let $(p,q) \in \Delta^\complement$, then $p \ne q$. Since $X$ is Hausdorff, there 
    exists respective open neighborhoods $U,V$ of $p,q$ such that $U \cap V = \emptyset$. Since both $U,V$ are open
    their cartesian product is open in the product space, and is contained in $\Delta^\complement$ since they are 
    disjoint. Therefore $\Delta$ is closed. \\
    The other direction is the exact reasoning just reversed.
    \qed
\end{s}

\begin{p}{3.7}
    Show that the space $X$ of Problem $2-22$ is homeomorphic to $\mathbb{R}_d \times \mathbb{R}$ where 
    $\mathbb{R}_d$ is the set $\mathbb{R}$ with the discrete topology.
\end{p}
\begin{s}{3.7}
    We can show that the component spaces are homeomorphic. By definition the first component of the space 
    in problem 2-22 the element can be anything so it has the discrete topology. The second component is
    $a \le y \le b$ which is the same as $x < b-a = \epsilon$ for some $\epsilon >0$ which is the same as
    the euclidean topology.
    \qed
\end{s}

\begin{p}{3.8}
    Let $X$ denote the Cartesian product of countably infinite many copies of $\mathbb{R}$
    (which is just the set of all infinite sequence of real numbers), endowed with the box topology. 
    Define a map $f: \mathbb{R} \to X$ by $f(x)= (x,x,x,\dots)$.
    Show that $f$ is not continuous, even though each of its component functions is.
\end{p}
\begin{s}{3.8}
    Consider the cardinality of the bases of the domain and co-domain. We know $\mathbb{R}$ is second countable so
    it has a countable basis. On the other hand, the infinite sequence of countable sets is not countable (which
    can be shown by diagonalization argument). Since the codomain has strictly more open sets than the domain, 
    the function cannot be continuous.
    \qed
\end{s}

\begin{p}{3.9}
    Let $X$ be as in the preceding problem. Let $X^+ \subseteq X$ be the subset consisting of the sequences 
    of strictly positive real numbers, and let $z$ denote the zero sequence, that is, the one whose terms are 
    $z_i = 0$ for all $i$. Show that $z$ is in the closure of $X^+$, but there is no sequence of elements of
    $X^+$ converging to $z$. Then use the sequence lemma to conclude that $X$ is not first countable, and thus
    not metrizable.
\end{p}
\begin{s}{3.9}
    The zero sequence is in the closure because we can show that it is on the boundary. I.e. any open set that contains 
    the zero sequence also contains some point of $X^+$. This is due to the archimedian property of the reals.
    Any open interval containing $0$ will be have some positive epsilon, which we can fit an open interval of length less
    than epsilon. \\
    By definition a sequence is countable; and we also have an infinite countable number of copies of the reals from 
    which to pick open sets that do not include the $n$th term of the $n$th sequence. Then by construction there 
    does not exist an $N$ such that if $n \ge N$, $a_n \in V(z)$.
    \qed
\end{s}

\begin{p}{problem 3.10}
    Suppose that $(X_\alpha)_{\alpha \in A}$ is an indexed family of topological spaces, and $Y$ is an topological
    space. A map $f: \coprod_{\alpha \in A}X_\alpha \to Y$ is continuous if and only if its restriction to each 
    $X_\alpha$ is continuous. The disjoint union topology is the unique topology on $\coprod_{\alpha \in A}X_\alpha$ 
    with this property.
\end{p}
\begin{s}{problem 3.10}
    The gluing lemma shows that if each restriction is continuous, then $f$ is continuous.\\
    Assume that $f$ is continuous. Then the restriction to each $X_\alpha$ must also be continuous because
    each open set in $X_\alpha$ is also open in the disjoint union space since $X_\alpha$ is by definition open
    so any open set intersected with it is also open.\\
    Let thehe identity map between
    $\coprod_{\alpha \in A}X_\alpha$ and $\coprod_{\alpha \in A}X_\alpha$ with 2 topologies be $f$, assuming that $f$
    has the the characteristic property, we have that the restrictions to $X_\alpha$ with the subspace topology is 
    straightforwardly continuous. Thus $f$ is continuous and a homeomorphism.
    \qed
\end{s}


\begin{p}{3.11}
    Proposition 3.62(d) showed that the restriction of a quotient map to a saturated open subset is a quotient map 
    onto its image. Show that the "saturated" hypotehsis is necessary, by giving an example of a quotient map 
    $f: X \to Y$ and an open subset $U \subseteq X$ such that $f|_U :U \to Y$ is surjective but not a quotient map.
\end{p}
\begin{s}{3.11}
    TODO
\end{s}


\begin{p}{3.12}
    Suppose $X$ is a topological space and $(X_\alpha)_{\alpha \in A}$ is an indexed family of topological spaces.\\
    (a) For any subset $S \subseteq X$, show that the subspace topology on $S$ is the coursest topology for which
    $\iota_S: S \hookrightarrow X$ is continuous.\\
    (b) Show that the product topology is the coarsest topology on $\prod_{\alpha\in A}X_\alpha$ for which the 
    canonical projection $\pi_\alpha: \prod_{\alpha\in A}X_\alpha \to X_\alpha$ is continuous.\\
    (c) Show that the disjoint union topology is the finest topology on $\coprod_\alpha X_\alpha$ for which every
    canonical injection $\iota_\alpha : X_\alpha \to \coprod_\alpha X_\alpha$ is continuous.\\
    (d) Show that if $q: X \to Y$ is any surjective map, the quotient topology on $Y$ is the finest topology for which
    $q$ is continuous.
\end{p}
\begin{s}{3.12}
    (a) The least amount of open sets would need to be sets of $S$ for which the preimage of the inclusion map is open.
    In this context, the preimage is just the intersection $X \cap S$ which is precisely the definition of the subspace 
    topology.\\
    (b) same reasoning as above\\
    (c) inverse of the two above, if it were any finer, the original space $X_\alpha$ would not have enough open sets.\\
    (d) Follows from the definition of the quotient topology. If $q(U)$ is open, then $U$ is as well, by definition of 
    continuity, but also by definition of the quotient topology.
    \qed
\end{s}

\begin{p}{3.13}
    Suppose $X$ and $Y$ are topological spaces and $f: X \to Y$ is a continuous map. Prove the following:\\
    (a) If $f$ admits a continuous left inverse, it is a topological embedding.\\
    (b) If $f$ admits a continuous right inverse, it is a quotient map.\\
    (c) Give examples of a topological embedding with no continuous left inverse, and a quotient map 
    with no continuous right inverse.
\end{p}
\begin{s}{3.13}
    (a) Since $f$ admits a continuous left inverse, we have that $f$ is an open mapping; and when restricting the co-domain
    to the image, then we have a homeomorphism; thus $f$ is a topological embedding.\\
    (b) Since $f$ admits a continuous right inverse, we have that $f$ is surjective, since 
    $\forall y\in f(X), \exists x \in X \, s.t. \, x = f^{-1}(y)$, applying $f$, we have that 
    $f(x) = y$ for all $y$ in the image. Thus we have that $f$ is an open and continuous mapping that is surjective, 
    which is thus a quotient map.\\
    (c) Any embedding such that the co-domain strictly subsumes the image will work. \\
    The continuous function taking the the closed unit interval to the circle.
    \qed
\end{s}

\begin{p}{3.14}
    Show that the real projective space $\mathbb{P}^n$ is an $n$-manifold. [Hint: consider the subsets 
    $U_i \subseteq \mathbb{R}^{n+1}$ where $x_i=1$.]
\end{p}
\begin{s}{3.14}
    Following the hint, let $U_i = \{(x_1,\dots,x_{n+1}) \, | \, x_i \ne 0\}$. Then set of $U_i$ for $i=1,\dots,n+1$ is
    an open cover for $\mathbb{R}^{n+1}$. Then the images $V_i=\pi(U_i)$ cover $\mathbb{P}^n$.
    Then from each $V_i$, define a mapping $\phi_i: V_i \to \mathbb{R}^n$ by 
    $$\phi_i([x_1,\dots,x_{n+1}])=(\frac{x_1}{x_i},\dots,\frac{x_{n+1}}{x_i})$$
    Let ${\pi_i}= \pi_{|U_i}$ be the restriction of the canonical projection to $U_i$. Note that this is also a quotient
    map since $U_i$ is open.
    Note that the mapping $\psi_i: U_i \to \mathbb{R}^n$ defined by
    $\psi_i((x_1,\dots,x_{n+1})) = (\frac{x_1}{x_i},\dots,\frac{x_{n+1}}{x_i})$
    is continuous (similar to the construction of the stereographic projection). Also note that 
    $\psi_i = \phi_i \circ \pi_i$. This implies $\phi_i: V_i \to \mathbb{R}^n$ is continuous.\\
    From propostion 3.56, we have that $\mathbb{P}^n$ is second countable.\\
    For Hausdorffness, let $p,q$ be distrinct points in $\mathbb{P}^n$. 
    TODO
\end{s}

\begin{p}{3.15}
    TODO
\end{p}
\begin{s}{3.15}
\end{s}

\begin{p}{3.16}
    Let $X$ be the subset $(\mathbb{R}\times \{0\}) \cup (\mathbb{R}\times \{1\})$. Define an equivalence relation on 
    $X$ by declaring $(x,0)\sim (x,1)$ if $x\ne 0$. Show that the quotient space $X/\sim$ is locally Euclidean and 
    second countable, but not Hausdorff. (This space is called the line with two origins).
\end{p}
\begin{s}{3.16}
    The subsets $\{(x,0)\}$ and $\{(x,1)\}$ are homeomorphic to $\mathbb{R}$, and also passes to the quotient, therefore 
    the line with two origins is locally euclidean. By proposition 3.56, it is also second countable.\\
    However, it is not Hausdorff because any two open sets containing the origins must overlap. This is because for any 
    $\epsilon > 0$, $(\epsilon,0)\sim (\epsilon,1)$. 
    \qed
\end{s}

\begin{p}{3.17}
    This problem show that the conclusion of proposition 3.57 need not be true if the quotient map is not assumed to be 
    open. Let $X$ be the following subset of $\mathbb{R}^2$:
    $$X = ((0,1)\times (0,1))\cup \{(0,0)\} \cup \{(1,0)\}$$
    For any $\epsilon \in (0,1)$, let $C_\epsilon$ and $D_\epsilon$ be the sets
    $$C_\epsilon = \{(0,0)\} \cup ((0,\frac{1}{2})\times (0,\epsilon))$$
    $$D_\epsilon = \{(1,0)\} \cup ((\frac{1}{2},1)\times (0,\epsilon))$$
    Define a basis $\mathcal{B}$ for a topology on $X$ consisting of all open rectangles of the form
    $(a_1,b_1)\times (a_2,b_2)$ with $0\le a_1 < b_1 \le 1$ and $0\le a_2 < b_2 \le 1$, together 
    with all subsets of the form $C_\epsilon$ or $D_\epsilon$.\\
    (a) Show that $\mathscr{B}$ is a basis for a topology on $X$.\\
    (b) Show that this topology is Hausdorff.\\
    (c) Show that the subset $A=\{(0,0)\} \cup ((0,\frac{1}{2}]\times (0,1))$ is closed in $X$.\\
    (d) Let $\sim$ be the relation on $X$ generated by $a \sim a'$ for all $a,a' \in A$. Show that $\sim$ is closed 
    in $X\times X$.\\
    (e) Show that the quotient space $X/A$ obtained by collapsing $A$ to a point is not Hausdorff.
\end{p}
\begin{s}{3.17}
    (a) It is clear that any point in $X$ is also in an element of the basis.
    Let $B_1, B_2 \in \mathscr{B}$, and let $p\in X$ such that $p \in B_1 \cap B_2$. Want to show that 
    $\exists B_3 \in \mathscr{B} \subseteq B_1 \cap B_2$ such that $p\in B_3$.
    If both $B_1$ and $B_2$ are open rectangles, it is clear that their intersection is also an open rectangle.\\
    WLOG assume that $B_1$ is an open rectangle. Then if $B_2$ is either a $C_\epsilon$ or $D_\epsilon$, then 
    the intersection must also be an open rectangle with $x$ axis to be either lower or upper half.\\
    If both are $C_\epsilon$s or $D_\epsilon$s, then the intersection will be $C_\epsilon$ or $D_\epsilon$ respectively
    for some other epsilon (min). \\
    Finally, assume $B_1=C_{\epsilon_1}$ and $B_2=D_{\epsilon_2}$. Then the intersection is empty. 
    Thus $\mathscr{B}$ is a basis for some topology.\\
    (b) Let $p,q \in X$ be distinct point of $X$. 
    If $p$ is $(0,0)$ or $(1,0)$, and $q$ is not, then some open rectangle can be chosen that is disjoint from $D$ or $C$
    since $\epsilon$ can also be chosen to make make the half rectangle arbitrarily thin.\\
    Otherwise, open rectangles separate each point in the open unit rectangle.\\
    (c) $A^\complement = \{(1,0)\} \cup ((\frac{1}{2},1)\times (0,1))$, which is open since it is 
    $\bigcup_{\epsilon \in (0,1)} D_\epsilon$. Thus $A$ is closed. \\
    (d) $\sim$ is equal to $A\times A$. Since $A$ is closed, the product is closed.\\
    (e) Consider the two points $A$ and $(1,0)$ in the quotient space. $A$ is not open so any open set containing it
    in the original space would have to be the union of $\bigcup_{\epsilon \in (0,1)} C_\epsilon$
    and some open rectangle in $(\frac{1}{2},k)\times (0,1)$. However, any open set containing $(1,0)$ will intersected
    any open set in $(\frac{1}{2},k)\times (0,1)$, thus the quotient is not Hausdorff
    \qed
\end{s}

\begin{p}{3.18}
    Let $A\subseteq \mathbb{R}$ be the set of integers, and let $X$ be the quotient space $\mathbb{R} / A$
    obtained by collapsing $A$ to a point as in Example 3.52. (We are not using the notation $\mathbb{R}/\mathbb{Z}$
    for this space because that has a different meaning, described in Example 3.92.)\\
    (a) Show that $X$ is homeomorphic to a wedge sum of countably infinitely many circles. [Hint: express both 
    spaces as a quotient of disjoint union of intervals.]\\
    (b) Show that the equivalence class $A$ does not have a countable neighborhood basis in $X$, so $X$ is not 
    first or second countable.
\end{p}
\begin{s}{3.18}
    (a) $\mathbb{R}$ can be expressed as $\cup_{\alpha \in A} I_{\alpha,\alpha+1}$, i.e. unit intervals indexed
    by the set of integers. Note that the ends points are exactly the elements of $A$, by construction. 
    We can first identify the endpoints of each interval with each other, i.e. 
    $\cup_{\alpha \in A} I_{\alpha,\alpha+1} / \{\alpha,\alpha+1\}$. Then notice that all terms of the union are 
    disjoint, thus it is homeomorphic to $\coprod_{\alpha \in A} I_{\alpha,\alpha+1} / \{\alpha,\alpha+1\}$.
    Then identify all the $\{\alpha,\alpha+1\}$ in each term,
    $$(\coprod_{\alpha \in A} I_{\alpha,\alpha+1} / \{\alpha,\alpha+1\})/{[A]}$$
    where $[A]$ is the set $A$ in the quotient. Thus we have collapsed the original set $A$ to a countably infinite
    wedge sum of circles.\\
    (b) Assume to the contrary that the point $[A]$ in the quotient has a countable neighborhood basis. Thus 
    there exists $U_1,U_2,U_3,\dots$ such that every open neighborhood of $[A]$ is contained in one of the basis listed.
    An open neighborhood of $[A]$ will be an open interval around some point of $A$ in each term of the disjoint sum in
    the preimage of the quotient. Then construct an open neighborhood of $[A]$ by choosing for each $U_i$, an interval
    that is strictly contained within the $i$th term of the disjoint sum's interval. Then by construction, our open 
    neighborhood $U$ is not contained in any $U_i$, thus contradicting the assumption that the quotient space is 
    first countable.
    \qed
\end{s}

\begin{p}{3.19}
    Let $G$ be a topological group and let $H \subseteq G$ be a subgroup. Show that its closure $\overline{H}$ 
    is also a subgroup.
\end{p}
\begin{s}{3.19}
    Notice that the binary op in the next problem looks like the subgroup criterion. Using the result in problem 3.20,
    we have that $f: G \times G \to G$ given by $(x,y) \mapsto xy^{-1}$ is continuous. \\
    Since $f$ is continuous, and since $\overline{H}$ is closed, the preimage of $f^{-1}(\overline{H})$
    is also closed. Then we have that $\overline{H\times H} \subseteq f^{-1}(\overline{H})$.
    Using the fact that $\overline{H}\times\overline{H} = \overline{H\times H}$, 
    we have that $\overline{H}\times\overline{H} \subseteq \overline{H\times H} \subseteq f^{-1}(\overline{H})$.
    Applying $f$ to each set we get 
    $$f(\overline{H}\times\overline{H}) \subseteq f(\overline{H\times H}) \subseteq \overline{H}$$
    By the subgroup criterion, we have that $\overline{H}$ is a subgroup of $G$.
    \qed
\end{s}

\begin{p}{3.20}
    Suppose $G$ is a group that is also a topological space. Show that $G$ is a topolgical group if and only if the
    map $G \times G \to G$ given by $(x,y) \mapsto xy^{-1}$ is continuous.
\end{p}
\begin{s}{3.20}
    Assume $G$ is a topological group. Since both the identity map and inverse maps are continuous, we have that
    the map $G\times G \to G\times G$ given by $(x,y) \mapsto (id(x), y^{-1})=(x,y^{-1})$ is continuous. Then composing
    such a map with the multiplication map, which is also continuous, we have that 
    $$(x,y)\mapsto (x,y^{-1}) \mapsto xy^{-1}$$
    is continuous.\\
    Assume that the map given by $(x,y) \mapsto xy^{-1}$ is continuous. Since this applies for all $x \in G$, we can 
    fix $x=e$, that is the identity element in the group $G$. Then we have that for all $y\in G$, 
    $(e,y) \mapsto ey^{-1}= y^{-1}$, which is homeomorphic to the map on $G\to G$ given by $y \mapsto y^{-1}$, 
    that is the inverse map on $G$ is continuous. Then define a map similarly as above, using the fact that 
    the identity map and the inverse maps are continuous, to get that 
    $$(x,y) \mapsto (x,y^{-1}) \mapsto x(y^{-1})^{-1} = xy$$
    is continuous.
    \qed
\end{s}

\begin{p}{3.21}
    Let $G$ be a topological group and $\Gamma \subseteq G$ be a subgroup.\\
    (a) For each $g\in G$, show that there is a homeomorphism $\theta_g : G / \Gamma \to G / \Gamma$
    such that the folliwing diagram commutes:
    \[
    \begin{tikzcd}
        G \arrow{d}{} \arrow{r}{L_g} & G \arrow{d}{} \\
        G / \Gamma \arrow{r}{\theta_g}  & G / \Gamma
    \end{tikzcd}
    \]
    (b) Show that every coset space is topologically homogeneous.
\end{p}
\begin{s}{3.21}
    (a) Applying $L_g$ then quotienting is $q_\Gamma \circ L_g(x) = (gx)\Gamma$. 
    Then it is clear that our $\theta_g : g / \Gamma \to g / \Gamma$
    should be $\theta_g(x\Gamma) = (gx)\Gamma$ where $gx$ is the group multiplication.
    This mapping is continuous since the group multiplication is continuous. And the continuous inverse is given by 
    $\theta_{g^{-1}}$ since 
    $$\theta_{g^{-1}}(\theta_g(x\Gamma)) = (g^{-1}(gx))\Gamma = ((g^{-1}g)x)\Gamma = x\Gamma$$
    (b) Every coset is topologically homogenous since each coset is formed by $g\Gamma$ for $g\in G$. Since $G$ is a group,
    for any $x\in G, \exists g' \in G$ such that $g'g = x$. Then we can just apply $\theta_g'$ which is a homeomorphism.
    \qed
\end{s}

\begin{p}{3.22}
    Let $G$ be a group acting by homeomorphisms on a topological space $X$, and let $\mathscr{O} \subseteq X\times X$
    be the subset defined by
    $$\mathscr{O} = \{(x_1,x_2):x_1=g\cdot x_2 \text{for some } g \in G\}$$
    (a) Show that the quotient map $X\to X / G$ is an open map.\\
    (b) Conclude that $X/G$ is Hausdorff if and only if $\mathscr{O}$ is closed in $X\times X$.
\end{p}
\begin{s}{3.22}
    (a) Let $U\subseteq X$ be open. Then we have that $q(U)=\{[O_x] \text{for }x \in U\}$, that is the orbits are 
    the equivalence classes in the quotient space. $q^{-1}(q(U)) = \bigcup_{g\in G} g(U)$ since $G$ is an action
    by homeomorphisms, each $g(U)$ is open, so the union is open. Thus since $q^{-1}(q(U))$ is open,
    thus $q^(U)$ is open (in quotient topology), thus $q$ is an open mapping.\\
    (b) Since $q$ is an open mapping, the corollary 3.58 applies.
    \qed
\end{s}

\begin{p}{3.23}
    Suppose $\Gamma$ is a normal subgroup of the topological group $G$. Show that the quotient group $G/\Gamma$ is 
    a topological group with the quotient topology. [Hint: it might be helpful to use Problems 3-5 and 3.22.]
\end{p}
\begin{s}{3.23}
    Consider the commutative diagram
    \[
    \begin{tikzcd}
        G \times G \arrow{d}{Q} \arrow{r}{m} & G \arrow{d}{q} \\
        G/\Gamma \times G/\Gamma \arrow{r}{M}  & G / \Gamma
    \end{tikzcd}
    \]
    where $q$ is the canonical quotient map, $m$ the continuous inversion map $m(x,y) = xy^{-1}$, 
    $M(x\Gamma,y\Gamma) = (xy^{-1})\Gamma$, and $Q((x,y)) = (q(x),q(y))$.\\
    Need to show that $M$ is continuous. It is clear that $q \circ m$ is continuous.
    The commutative diagram now looks like
    \[
    \begin{tikzcd}
        G \times G \arrow{d}{Q} \arrow{dr}{q \circ m} \\
        G/\Gamma \times G/\Gamma \arrow{r}{M}  & G / \Gamma
    \end{tikzcd}
    \]
    Now since $q$ is an open map (from problem 3.22), and by 3.23 the product is also an open map. Which means
    $Q$ is a quotient map. Thus the characteristic property of quotient maps applies, and since $q\circ m$ is 
    continuous, $M$ is continuous.
    \qed
\end{s}

\begin{p}{3.24}
    Consider the action of $O(n)$ on $\mathbb{R}^n$ by matrix multiplication as in Example 3.88(b). 
    Prove that the quotient space is homeomorphic to $[0,\infty)$. [Hint: consider the function
    $f: \mathbb{R}^n \to [0,\infty)$ given by $f(x)=|x|$.]
\end{p}
\begin{s}{3.24}
    From example 3.88b, we know that the orbits of this group action is the spheres centered at 0 and the origin.
    Following the hint, the mapping from an element to its norm is continuous. Now we just need to construct a 
    continuous inverse, $g: [0,\infty) \to \mathbb{R}^n/O(n)$. Consider the embedding
    $x \mapsto (x,\dots,0)$ from $[0,\infty) \to \mathbb{R}^n$. It is clearly continuous. 
    Then quotienting by $O(n)$ gives us our continuous inverse.
    \qed
\end{s}

\end{document}
