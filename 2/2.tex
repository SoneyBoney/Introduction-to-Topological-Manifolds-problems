%%%%%%%%%%%%%%%%%%%%%%%%%%%%%%%%%%%%%%%%%%%%%%%
%%%This is a science homework template. Modify the preamble to suit your needs. 
%The junk text is   there for you to immediately see how the headers/footers look at first 
%typesetting.


\documentclass[12pt]{article}

%AMS-TeX packages
\usepackage{amssymb,amsmath,amsthm} 
%geometry (sets margin) and other useful packages
\usepackage[margin=1.25in]{geometry}
\usepackage{graphicx,ctable,booktabs,mathrsfs}


%
%Redefining sections as problems
%
\makeatletter
\newenvironment{p}{\@startsection
       {section}
       {1}
       {-.2em}
       {-3.5ex plus -1ex minus -.2ex}
       {2.3ex plus .2ex}
       {\pagebreak[3]%forces pagebreak when space is small; use \eject for better results
       \large\bf\noindent{P }
       }
       }
\makeatother

\makeatletter
\newenvironment{s}{\@startsection
       {subsection}
       {2}
       {-.2em}
       {-3.5ex plus -1ex minus -.2ex}
       {2.3ex plus .2ex}
       {\pagebreak[3]%forces pagebreak when space is small; use \eject for better results
       \large\bf\noindent\emph{(sol) }
       }
       }
\makeatother

%
%Fancy-header package to modify header/page numbering 
%
\usepackage{fancyhdr}
\pagestyle{fancy}
%\addtolength{\headwidth}{\marginparsep} %these change header-rule width
%\addtolength{\headwidth}{\marginparwidth}
\chead{} 
\rhead{\thepage} 
\cfoot{} 
\renewcommand{\headrulewidth}{.3pt} 
\renewcommand{\footrulewidth}{.3pt}
\setlength\voffset{-0.25in}
\setlength\textheight{648pt}

%%%%%%%%%%%%%%%%%%%%%%%%%%%%%%%%%%%%%%%%%%%%%%%

%
%Contents of problem set
%    
\begin{document}

\title{Chapter 2 Problems}
\author{Campinghedgehog}
\date{August 5, 2023}

\maketitle

\thispagestyle{empty}


\begin{p}{1}
    Let $X$ be an infinite set.\\
    (a) Show that 
    $$\mathscr{T}_1 = \{U\subseteq X : U = \emptyset \text{ or } X \setminus U \text{ is finite } \}$$
    is a topology on $X$, called the finite complement topology.\\
    (b) Show that 
    $$\mathscr{T}_2 = \{U\subseteq X : U = \emptyset \text{ or } X \setminus U \text{ is countable } \}$$
    is a topology on $X$, called the countable complement topology.\\
    (c) Let $p$ be an arbitrary point in $X$, and show that 
    $$\mathscr{T}_3 = \{U\subseteq X : U = \emptyset \text{ or } p\in U \}$$
    is a topology on $X$, called the particular point topology.\\
    (d) Let $p$ be an arbitrary point in $X$, and show that 
    $$\mathscr{T}_4 = \{U\subseteq X : U = X \text{ or } p \notin U \}$$
    is a topology on $X$, called the excluded point topology.\\
    (e) Determine whether
    $$\mathscr{T}_5 = \{U\subseteq X : U = \emptyset \text{ or } X \setminus U \text{ is infinite } \}$$
    is a topology on $X$. 
\end{p}
\begin{s}{1}
    (a) By definition $\emptyset \in \mathscr{T}_1$. But also $X$ is open since $\emptyset$ is a finite set.
    Let $U_1$ and $U_2$ be open sets. Then $X \setminus (U_1 \cap U_2) = (X\setminus U_1) \cup (X\setminus U_2)$,
    both of which are finite which means their union is finite. Induction shows that finite intersection of open sets is open.
    Let $\bigcup_{\alpha \in A} U_{\alpha}$ be a union of open sets. Then
    $X \setminus\bigcup_{\alpha \in A} U_{\alpha} = \bigcap_{\alpha \in A}(X \setminus U_\alpha)$.
    Since each $(X \setminus U_\alpha)$ is finite, the intersection is finite. \\
    (b) The prove for countable complement is exactly the same as finite since all the same conditions apply for countablility.\\
    (c) Empty set and $X$ are open by definition. Finite intersection of sets containing $p$ will also contain $p$. Same for arbitrary union.\\
    (d) $X$ is open by definition. $\emptyset$ is open since it contains no points. The rest is the same as (c)\\
    (e) No. Unions can become non infinite. Eg. Integers, even numbers and odd numbers. Complement is empty.
\end{s}


\begin{p}{2}
    Let $X=\{1,2,3\}$. Give a list of topologies on $X$ such that every topology on $X$ is homeomorphic to exactly one on your list.
\end{p}
\begin{s}{2}
    TODO
\end{s}


\begin{p}{3}
    Let $X$ be a topological space and $B$ be a subset of $X$. Prove that following set equalities.\\
    (a) $\overline{X \setminus B} =  X \setminus \, Int \, B$ \\
    (b) $Int(X \setminus B) = X\setminus \overline{B}$
\end{p}
\begin{s}{3}
    (a) $\overline{X \setminus B}$ is the exterior of $B$ plus boundary, which is equivalent to $X$ minus the interior of $B$.\\
    (b) $Int(X \setminus B)$ is equal to the exterior of $B$, which is equal by definition to $X\setminus \overline{B}$.
\end{s}

\begin{p}{4}
    Let $X$ be a topological space and $\mathscr{A}$ be a collection of subsets of $X$. Prove the
    following containments:\\
    (a) $$\overline{\bigcap_{A \in \mathscr{A}} A} \subseteq \bigcap_{A\in \mathscr{A}}\overline{A}$$
    (b) $$\overline{\bigcup_{A \in \mathscr{A}} A} \supseteq \bigcup_{A\in \mathscr{A}}\overline{A}$$
    (c) $$Int\bigg(\bigcap_{A \in \mathscr{A}}A\bigg)\subseteq \bigcap_{A \in \mathscr{A}} Int \, A$$
    (d) $$Int\bigg(\bigcup_{A \in \mathscr{A}}A\bigg)\supseteq \bigcup_{A \in \mathscr{A}} Int \, A$$
    When $\mathscr{A}$ is a finite collection, show that the equality holds in (b) and (c), but not 
    necessarily in (a) or (d).
\end{p}
\begin{s}{4}
    (a) The righthand side is an intersection of closed sets containing $A$, and thus is closed. The 
    left hand side the the closure of the intersection of $A$s, which is the smallest closed set containing
    $A$s so the containment follows.\\
    (b) for all $A\in\mathscr{A}, A \subseteq \bigcup_{A \in \mathscr{A}} A$, which implies
    $$A\in\mathscr{A}, \overline{A} \subseteq \overline{\bigcup_{A \in \mathscr{A}} A}$$,
    which implies the resulting containment.\\
    (c) $$\forall A \in \mathscr{A}, \bigcap_{A \in \mathscr{A}}A \subseteq A$$
    $$\implies \forall A \in \mathscr{A}, Int\bigg(\bigcap_{A \in \mathscr{A}}A\bigg) \subseteq Int\,A$$
    Since this is for all $A$, the result follows.\\
    (d) The lefthand side is a union of open sets contained in $A$, and the right hand side
    is the largest open set contained in union $A$, so the result follows.\\
    When $\mathscr{A}$ is a finite collection, the unions of closures and intersections of interiors
    is closed and open respectively. And closed sets are equal to its closure and open sets its interior.
    TODO: idk about (a) and (b)
\end{s}

\begin{p}{5}
    Too hard
\end{p}
\begin{s}{}
\end{s}

\begin{p}{6}
    Prove Proposition 2.30(characterization of continuity, openness, and closedness in terms of closures and interiors).\\
    Suppose $X$ and $Y$ are topological spaces, and $f: X \to Y$ is any map.\\
    (a) $f$ is continuous if and only if $f(\overline{A}) \subseteq \overline{f(A)}$ for all $A \subseteq X$.\\
    (b) $f$ is closed if and only if $f(\overline{A}) \supseteq \overline{f(A)}$ for all $A \subseteq X$.\\
    (c) f is continuous if and only if $f^{-1}(\text{Int} \, B) \subseteq \text{Int} \, f^{-1}(B)$ for all $B \subseteq Y$.\\
    (d) f is open if and only if $f^{-1}(\text{Int }B) \supseteq \text{Int }f^{-1}(B)$ for all $B \subseteq Y$.
\end{p}
\begin{s}{6}
    (a) Let $B \subseteq Y$ be closed. Then $B = \overline{B}$, and also $f^{-1}(B) = f^{-1}(\overline{B})$. There exists 
    $A \subseteq X$ such that $A = f(B)$. Then $f(\overline{A}) \subseteq \overline{f(A)}$ implies 
    $\overline{A} \subseteq f(\overline{B})$. Then $f^{-1}(B) \subseteq \overline{A} \subseteq f^{-1}(\overline{B})$, and since
    $f^{-1}(B) = f^{-1}(\overline{B})$, $\overline{A} = f^{-1}(B)$ so $f^{-1}(B)$ is closed.\\
    Let $f$ be continuous. Let $f(A) \subseteq Y$, so $\overline{f(A)}$ is closed. Since $f$ is continuous,
    $f^{-1}(\overline{f(A)})$ is closed and contains $A$, thus it contains $\overline{A}$ since the closure of $A$ is the smallest
    closed set that contains $A$.\\
    (b) The proof is basically the same as part (a).\\
    (c) Let $B \subseteq Y$. Since $f$ is continuous, $f^{-1}(\text{Int }B)$ is open. And since $\text{Int }B \subseteq B$,
    $f^{-1}(\text{Int }B)$ is contained in $f^{-1}(B)$, so is contained in $\text{Int }f^{-1}(B)$.\\
    Conversely, let $B \subseteq Y$ be open. Then $B = \text{Int }B$. Then we have 
    $f^{-1}(B) \subseteq = f^{-1}(\text{Int }B) \subseteq \text{Int }f^{-1}(B)$. By definition, interiors are contained in the 
    subset, thus $\text{Int }f^{-1}(B) = f^{-1}(B)$, so the preimage is open so $f$ is continuous.\\
    (d) same as part (c)
\end{s}

\begin{p}{7}
    Prove proposition 2.39 (in a Hausdorff space, every neighborhood of a limit point contains
    infinitely many points of the set).
\end{p}
\begin{s}{7}
    Let $X$ be Hausdorff. Let $p \in X$ be a limit point. Suppose there exists some neighborhood of $p$
    with only finitely many points. Since $X$ is Hausdorff, we can separate any point in the 
    neighborhood (not equal to $p$ itself) with open sets. Then there exists some new neighborhood
    of $p$ that does not contain any of the points in the original neighborhood, which means there
    exists a neighborhood of $p$ that does not contain any other points, which contradicts the assumption
    that $p$ is a limit point.
    \qed
\end{s}


\begin{p}{8}
    Let $X$ be a Hausdorff space, let $A \subseteq X$, and let $A'$ denote the set of
    limit points of $A$. Show that $A'$ is closed in X.
\end{p}
\begin{s}{8}
    Will show that $X \setminus A'$ is open. Let $p\in X \setminus A'$. Let $V_p$ be an 
    open neighborhood of $p$ such that $V_p \cap A$ is empty or only contains $p$ itself.
    This exists because $p$ is not a limit point of $A$. With Hausdorff property, and since
    $p$ is an isolated point and thus $\{p\}$ is open, we have that $V_p \setminus \{p\}$
    is also open. By construction, $V_p \setminus \{p\} \cap A$ is empty, which means 
    all the points in $V_p$ are also not limit points of $A$ thus $V_p \subseteq X \setminus A'$,
    so $A'$ is closed.
    \qed
\end{s}

\begin{p}{9}
    Suppose $D$ is a discrete space, $T$ is a space with the trivial topology, $H$ is a 
    Hausdorff space, and $A$ is an arbitrary topological space.\\
    (a) Show that every map from $D$ to $A$ is continuous.\\
    (b) Show that every map from $A$ to $T$ is continuous.\\
    (c) Show that the only continuous map from $T$ to $H$ are the constant maps.
\end{p}
\begin{s}{9}
    (a) Every possible subset in a discrete space is open, so any preimage is open,
    so the funciton has to be continuous.\\
    (b) The empty preimage is trivially open. And the preimage of the whole domain is the 
    whole codomain which is open.\\
    (c) Suppose for the sake of contradiction that there exists a continuous map $f: T \to H$
    such that $f$ is not constant. Let $V$ be an open set that contains more than 2 points,
    call them $p,q$. Since $f$ is not constant we can let $f^{-1}(p)\ne f^{-1}(q)$, 
    Then we can separate the 2 points with disjoint open sets. It follows that the preimages of
    these 2 open sets are not both empty, nor both the whole co-domain. Thus contradicting the
    assumption that $T$ is a trivial space. 
    \qed
\end{s}

\begin{p}{10}
    Suppose $f,g: X \to Y$ are continuous maps and $Y$ is Hausdorff. Show that the set
    $\{x \in X : f(x)=g(x)\}$ is closed in $X$. Give a counterexample if $Y$ is not Hausdorff.
\end{p}
\begin{s}{10}
    Will show that $\{x \in X: f(x) \ne g(x)\}$ is open. Let $x\in \{x \in X: f(x) \ne g(x)\}$.
    Since $Y$ is Hausdorff, there exists disjoint open neighborhoods $V_{f(x)}$ and $V_{g(x)}$.
    Since both $f,g$ are continuous, the preimages $f^{-1}(V_{f(x)}),g^{-1}(V_{g(x)})$, are also
    open. Call them $U_f$ and $U_g$. $U_f \cap U_g$ contains $x$ by construction and since both
    are open, the intersection is open. The last step is to show that 
    $$U_f \cap U_g \subseteq \{x \in X: f(x) \ne g(x)\}$$
    Let $p \in U_f \cap U_g$ be arbitrary. Then since $V_{f(x)}$ and $V_{g(x)}$ are disjoint,
    $f(p) \ne g(p)$.\\
    If $Y$ is not Hausdorff, then the last part of the proof above may not be true.
    \qed
\end{s}

\begin{p}{11}
    Let $f:X\to Y$ be a continuous map between topologocial spaces, and let $\mathscr{B}$
    be a basis of the topology on $X$. Let $f(\mathscr{B})$ denote the collection
    $\{f(B):B \in \mathscr{B}\}$. Show that $f(\mathscr{B})$ is a basis for the topology 
    of Y if and only if $f$ is surjective and open. 
\end{p}
\begin{s}{11}
    $(\implies)$ Assume $f(\mathscr{B})$ is a basis for the topology of $Y$. Then $f$ is 
    surjective since a basis covers the entire space $Y$. If $f$ was not an open mapping,
    then there exists some $B\in \mathscr{B}$ such that $f(B)$ is not open. But this is not 
    possible since by definition of a basis, of which $f(\mathscr{B})$ is, each element 
    must be open. \\
    $(\impliedby)$ Assume that $f$ is surjective and open. Since $f$ is open, each 
    $f(B)$ is also open, fulfilling part of the definition of a basis. Now we just need
    to show that each open set in $Y$ is a union of some $f(B)s$. Here we probably use 
    the contiuity assumption. Let $V \subseteq Y$ be an open set. Since $f$ is continuous,
    $f^{-1}(V)$ is also open. Since $\mathscr{B}$ is a basis for $X$, there exists some 
    collection $A$ such that $f^{-1}(V)=\bigcup_{\alpha \in A} B_\alpha$. 
    So $V=f(f^{-1}(V))=f(\bigcup_{\alpha \in A} B_\alpha) \subseteq f(\mathscr{B})$.
    Thus $f(\mathscr{B})$ is a basis for $Y$.
    \qed
\end{s}

\begin{p}{12}
    Suppose $X$ is a set, and $\mathscr{A} \subseteq \mathscr{P}(X)$ is any collection of 
    subsets of $X$. Let $\mathscr{T} \subseteq \mathscr{P}(X)$ be the collection of subsets
    consisting of $X,\emptyset$, and all unions of finite intersection of elements of $\mathscr{A}$.\\
    (a) Show that $\mathscr{T}$ is a topology. (It is called the topology generated by $\mathscr{A}$,
    and $\mathscr{A}$ is called the subbasis for $\mathscr{T}$)\\
    (b) Show that $\mathscr{T}$ is the coursest topologt for which all the sets in $\mathscr{A}$ are open.\\
    (c) Let $Y$ be any topological space. Show that a map $f: Y\to X$ is continuous if and only if
    $f^{-1}(U)$ is open in $Y$ for every $U \in \mathscr{A}$.
\end{p}
\begin{s}{12}
    (a) By definition, $X,\emptyset$ are in $\mathscr{T}$. A finite intersection cannot 
    break out of $\mathscr{T}$ because intersections can only make sets smaller.
    So picking just single sets out of arbitrary unions, and then taking their intersection is
    the best way to break out. However, by definition $\mathscr{T}$ is formed by unions of 
    finite intersections, so by definition finite intersections of $\mathscr{T}$ are still in
    $\mathscr{T}$. An arbitrary union of elements of $\mathscr{T}$ is still an arbitrary union of
    of finite intersections of $A$. \\
    (b) If any set is missing from $\mathscr{T}$, then it is no longer a topology, which means that
    $\mathscr{T}$ is the coarsest topology for which all sets in $\mathscr{A}$ is open.\\
    (c) $(\implies)$ Assume $f$ is continuous. By definition every $U$ is open so its preimage is open.\\
    $(\impliedby)$ Assume $f^{-1}(U)$ is open in $Y$ for every $U\in \mathscr{A}$.
    Every open set in $X$ is an arbitrary union of finite intersections. Since $f^{-1}(U)$ is open,
    arbitrary unions of finite intersections of $f^{-1}(U)$s is also open. Thus every open set in
    $X$, the preimage is also open.
    \qed
\end{s}

\begin{p}{13}
    Let $X$ be a totally ordered set. Give $X$ the order topology, which is the topology generated
    by the subbasis consisting of all sets of the following forms for $a\in X$:
    $$(a,\infty)=\{x\in X : x > a\},$$
    $$(-\infty,a)=\{x\in X : x < a\}$$
    (a) Show that each set of the form $(a,b)$ is open in $X$ and each set of the form
    $[a,b]$ is closed.\\
    (b) Show that $X$ is Hausdorff.\\
    (c) For any pair of points $a,b \in X$ with $a<b$, show that $\overline{(a,b)} \subseteq [a,b]$.\\
    (d) Show that the order topology on $\mathbb{R}$ is the same as the Euclidian topology.
\end{p}
\begin{s}{13}
    (a) Let $(a,b)$ be an arbitrary "open" interval in $X$. It can be formed by taking the intersections
    $(a,\infty) \cap (-\infty,b)$. $[a,b]$ is closed since it can be formed by
    $X \setminus ((-\infty,a) \cup (b,\infty))$, the latter part of which is open, so the set difference
    is closed.\\
    (b) Let $a\in X$. From the subbasis we can form open set: $(a,a)$, i.e. the singleton set
    containing just $a$ because it is the intersection of $(-\infty,a)$ and $(a,\infty)$.
    Thus $X$ is Hausdorff.\\
    (c) $[a,b]$ is closed from part (a), and it contains $(a,b)$ so the closure of $(a,b)$ is
    contained in $[a,b]$.\\
    (d) TODO!
\end{s}

\begin{p}{14}
    Prove Lemma 2.48 (Sequence lemma).\\
    Suppose $X$ is a first countable space, $A$ is any subset of $X$, and $x$ is any point of $X$.\\
    (a) $x \in \overline{A}$ if and only if $x$ is a limit of a sequence of points in $A$.\\
    (b) $x \in \text{Int }A$ if and only if every sequence in $X$ converging to $x$ is eventually
    in $A$.\\
    (c) $A$ is closed in $X$ if and only if $A$ contains every limit of every convergent sequence
    of point in $A$.\\
    (d) $A$ is open in $X$ if and only if every sequence in $X$ converging to
    a point of $A$ is eventually in $A$.
\end{p}
\begin{s}{14}
    (a) $(\implies)$ Assume $x\in \overline{A}$. If $x\in A$, then the constant sequence of $x$ itself
    is a sequence of points in $A$ that converges to $x$. If not, then $x$ has to be a limit
    point of $A$. By definition of a limit point of $A$, for all open neighborhoods $V_x$ of $x$,
    there exists a point $p \in A$ such that $p\in V_x$. Because the space is first countable,
    this applies to a countable nested-neighborhood of $x$. So we can take each $x$ in our nested
    progression as our sequence.\\
    $(\impliedby)$ Assume $x$ is a limit of a sequence of points in $A$. By definition of a limit
    of a sequence, there exists $(x_i)_{i=1}^{\infty}$ such that for all open neighborhoods
    $V_x$ of $x$, there exists some $N$ such that if $i\ge N$, then $x_i \in V_x$. So either
    $x\in A$ for all open neighborhoods of $x$, there exists some point of $A$ contained in
    that neighborhood, which is the definition of a limit point.\\
    (b) $(\implies)$ Assume $x\in \text{Int }A$. Let $(x_i)_{i=1}^{\infty}$ be a sequence 
    that converges to $x$. Since the interior is open, there exists some neighborhood 
    $V_x$ of $x$ that is contained in $\text{Int }A$. 
    Since $(x_i)_{i=1}^{\infty}$ converges to $x$, there exists some $N$ such that 
    if $i \ge N$, then $x_i \in V_x$. Thus the sequence is eventually in $A$ since $A$ contains
    its interior.\\
    $(\impliedby)$ Similary, if every sequence converging to $x$ is eventually in $A$, then 
    there exists an open neighborhood of $x$ that is contained in $A$, thus the interior being the
    largest open set contained in $A$ also contains such a neighborhood.\\
    (c) If $A$ is closed then $A = \overline{A}$ so then part (a) applies. Conversely, if part 
    (a) applies then all points in $A$ is also in $\overline{A}$ so $A=\overline{A}$ so $A$ is closed.\\
    (d) Same proof as part (c)
    \qed
\end{s}

\begin{p}{15}
    Let $X$ and $Y$ be topological spaces.
    (a) Suppose $f: X \to Y$ is continuous and $p_n \to p$ in $X$. 
    Show that $f(p_n) \to f(p)$ in $Y$.\\
    (b) Prove that if $X$ is first countable, the converse is true:
    if $f: X\to Y$ is a map such that $p_n \to p$ in $X$ implies $f(p_n) \to f(p)$ in $Y$,
    then $f$ is continuous.
\end{p}
\begin{s}{15}
    (a) Let $V_p$ be an arbitrary open neighborhood of $f(p)$. Then $f^{-1}(V_p)$ is
    an open neighborhood of $p$. Since $p_n \to p$, there exists some $N$ such that if
    $i \ge N$, $p_i \in f^{-1}(V_p)$, thus $f(p_i) \in V_p$.\\
    (b) Assume $X$ is first countable and that 
    $f: X\to Y$ is a map such that $p_n \to p$ in $X$ implies $f(p_n) \to f(p)$ in $Y$.
    Let $V \subseteq Y$ be open. Want to show $f^{-1}(V)$ is open.
    Let $p\in f^{-1}(V)$ and let $p_n \to p$ be an arbitrary sequence in $X$ converging to $p$.
    By assumption, $f(p_n) \to f(p)$ in $Y$. Since $V$ is open, $f(p_n)$ is eventually in $V$.
    This means that $p_n$ is eventually in $f^{-1}(V)$ which implies $f^{-1}(V)$ is open.
    Thus $f$ is continuous. 
    \qed
\end{s}

\begin{p}{16}
    Let $X$ be a second countable topological space. Show that every collection of disjoint
    open subsets of $X$ is countable.
\end{p}
\begin{s}{16}
    A collection of disjoint open subsets are contained in some basis. Since the basis must 
    be countable, the collection of disjoint open subets must also be countable.
\end{s}

\begin{p}{17}
    Let $\mathbb{Z}$ be the set of integers. Say that a subset $U \subseteq \mathbb{Z}$
    is symmetric if it satisfies the following condition:
    $$\text{for each n}\in \mathbb{Z}, n\in U \text{ if and only if } -n \in U$$
    Define a topology on $\mathbb{Z}$ by declaring a subset open if and only if it is symmetric.\\
    (a) Show that this is a topology.\\
    (b) Show that it is second countable.\\
    (c) Let $A$ be the subet $\{-1,0,1,2\} \subseteq \mathbb{Z}$, and determine the interior,
    boundary, closure, and limit points of $A$.\\
    (d) Is $A$ open in $\mathbb{Z}$? Is it closed?
\end{p}
\begin{s}{17}
    (a) $\emptyset$ is trivially open. $\mathbb{Z}$ is open since it contains all its negations.
    Let $U,V$ be open. Then $U \cap V$ must also be open since if $n \in U \cap V$ then $n \in U$
    and $n \in V$ which means $-n \in U$ and $-n \in V$. Arbitrary unions of open sets 
    is also open since if $n$ is contained in the union, then it must be contained in one 
    of the compenent sets, which mean $-n$ is also contained in that set which means $-n$ is
    contained in the union.\\
    (b) $\{\{i,-i\} : i \in \mathbb{N}\}$ forms a basis and is countable by construction.\\
    (c) The interior is the largest open set contained in it so 
    $\text{Int }A=\{-1,1,0\}$.\\
    Boundary= $\{2,-2\}$ ?\\
    Closure = $\{-1,0,1,2,-2\}$\\
    Limit points = $\{-1,0,1,2,-2\}$ ?\\
    (d) A is neither open nor closed since it is equal to neither its interior or closure.
\end{s}

\begin{p}{}
\end{p}
\begin{s}{}
\end{s}


\end{document}
