%%%%%%%%%%%%%%%%%%%%%%%%%%%%%%%%%%%%%%%%%%%%%%%
%%%This is a science homework template. Modify the preamble to suit your needs. 
%The junk text is   there for you to immediately see how the headers/footers look at first 
%typesetting.


\documentclass[12pt]{article}

%AMS-TeX packages
\usepackage{amssymb,amsmath,amsthm} 
%geometry (sets margin) and other useful packages
\usepackage[margin=1.25in]{geometry}
\usepackage{graphicx,ctable,booktabs,mathrsfs}


%
%Redefining sections as problems
%
\makeatletter
\newenvironment{p}{\@startsection
       {section}
       {1}
       {-.2em}
       {-3.5ex plus -1ex minus -.2ex}
       {2.3ex plus .2ex}
       {\pagebreak[3]%forces pagebreak when space is small; use \eject for better results
       \large\bf\noindent{P }
       }
       }
\makeatother

\makeatletter
\newenvironment{s}{\@startsection
       {subsection}
       {2}
       {-.2em}
       {-3.5ex plus -1ex minus -.2ex}
       {2.3ex plus .2ex}
       {\pagebreak[3]%forces pagebreak when space is small; use \eject for better results
       \large\bf\noindent\emph{(sol) }
       }
       }
\makeatother

%
%Fancy-header package to modify header/page numbering 
%
\usepackage{fancyhdr}
\pagestyle{fancy}
%\addtolength{\headwidth}{\marginparsep} %these change header-rule width
%\addtolength{\headwidth}{\marginparwidth}
\chead{} 
\rhead{\thepage} 
\cfoot{} 
\renewcommand{\headrulewidth}{.3pt} 
\renewcommand{\footrulewidth}{.3pt}
\setlength\voffset{-0.25in}
\setlength\textheight{648pt}

%%%%%%%%%%%%%%%%%%%%%%%%%%%%%%%%%%%%%%%%%%%%%%%

%
%Contents of problem set
%    
\begin{document}

\title{Chapter 2 Problems}
\author{Campinghedgehog}
\date{August 5, 2023}

\maketitle

\thispagestyle{empty}


\begin{p}{1}
    Let $X$ be an infinite set.\\
    (a) Show that 
    $$\mathscr{T}_1 = \{U\subseteq X : U = \emptyset \text{ or } X \setminus U \text{ is finite } \}$$
    is a topology on $X$, called the finite complement topology.\\
    (b) Show that 
    $$\mathscr{T}_2 = \{U\subseteq X : U = \emptyset \text{ or } X \setminus U \text{ is countable } \}$$
    is a topology on $X$, called the countable complement topology.\\
    (c) Let $p$ be an arbitrary point in $X$, and show that 
    $$\mathscr{T}_3 = \{U\subseteq X : U = \emptyset \text{ or } p\in U \}$$
    is a topology on $X$, called the particular point topology.\\
    (d) Let $p$ be an arbitrary point in $X$, and show that 
    $$\mathscr{T}_4 = \{U\subseteq X : U = X \text{ or } p \notin U \}$$
    is a topology on $X$, called the excluded point topology.\\
    (e) Determine whether
    $$\mathscr{T}_5 = \{U\subseteq X : U = \emptyset \text{ or } X \setminus U \text{ is infinite } \}$$
    is a topology on $X$. 
\end{p}
\begin{s}{1}
    (a) By definition $\emptyset \in \mathscr{T}_1$. But also $X$ is open since $\emptyset$ is a finite set.
    Let $U_1$ and $U_2$ be open sets. Then $X \setminus (U_1 \cap U_2) = (X\setminus U_1) \cup (X\setminus U_2)$,
    both of which are finite which means their union is finite. Induction shows that finite intersection of open sets is open.
    Let $\bigcup_{\alpha \in A} U_{\alpha}$ be a union of open sets. Then
    $X \setminus\bigcup_{\alpha \in A} U_{\alpha} = \bigcap_{\alpha \in A}(X \setminus U_\alpha)$.
    Since each $(X \setminus U_\alpha)$ is finite, the intersection is finite. \\
    (b) The prove for countable complement is exactly the same as finite since all the same conditions apply for countablility.\\
    (c) Empty set and $X$ are open by definition. Finite intersection of sets containing $p$ will also contain $p$. Same for arbitrary union.\\
    (d) $X$ is open by definition. $\emptyset$ is open since it contains no points. The rest is the same as (c)\\
    (e) No. Unions can become non infinite. Eg. Integers, even numbers and odd numbers. Complement is empty.
\end{s}


\begin{p}{2}
    Let $X=\{1,2,3\}$. Give a list of topologies on $X$ such that every topology on $X$ is homeomorphic to exactly one on your list.
\end{p}
\begin{s}{2}
    TODO
\end{s}


\begin{p}{3}
    Let $X$ be a topological space and $B$ be a subset of $X$. Prove that following set equalities.\\
    (a) $\overline{X \setminus B} =  X \setminus \, Int \, B$ \\
    (b) $Int(X \setminus B) = X\setminus \overline{B}$
\end{p}
\begin{s}{3}
    (a) $\overline{X \setminus B}$ is the exterior of $B$ plus boundary, which is equivalent to $X$ minus the interior of $B$.\\
    (b) $Int(X \setminus B)$ is equal to the exterior of $B$, which is equal by definition to $X\setminus \overline{B}$.
\end{s}

\begin{p}{4}
    Let $X$ be a topological space and $\mathscr{A}$ be a collection of subsets of $X$. Prove the
    following containments:\\
    (a) $$\overline{\bigcap_{A \in \mathscr{A}} A} \subseteq \bigcap_{A\in \mathscr{A}}\overline{A}$$
    (b) $$\overline{\bigcup_{A \in \mathscr{A}} A} \supseteq \bigcup_{A\in \mathscr{A}}\overline{A}$$
    (c) $$Int\bigg(\bigcap_{A \in \mathscr{A}}A\bigg)\subseteq \bigcap_{A \in \mathscr{A}} Int \, A$$
    (d) $$Int\bigg(\bigcup_{A \in \mathscr{A}}A\bigg)\supseteq \bigcup_{A \in \mathscr{A}} Int \, A$$
    When $\mathscr{A}$ is a finite collection, show that the equality holds in (b) and (c), but not 
    necessarily in (a) or (d).
\end{p}
\begin{s}{4}
    (a) The righthand side is an intersection of closed sets containing $A$, and thus is closed. The 
    left hand side the the closure of the intersection of $A$s, which is the smallest closed set containing
    $A$s so the containment follows.\\
    (b) for all $A\in\mathscr{A}, A \subseteq \bigcup_{A \in \mathscr{A}} A$, which implies
    $$A\in\mathscr{A}, \overline{A} \subseteq \overline{\bigcup_{A \in \mathscr{A}} A}$$,
    which implies the resulting containment.\\
    (c) $$\forall A \in \mathscr{A}, \bigcap_{A \in \mathscr{A}}A \subseteq A$$
    $$\implies \forall A \in \mathscr{A}, Int\bigg(\bigcap_{A \in \mathscr{A}}A\bigg) \subseteq Int\,A$$
    Since this is for all $A$, the result follows.\\
    (d) The lefthand side is a union of open sets contained in $A$, and the right hand side
    is the largest open set contained in union $A$, so the result follows.\\
    When $\mathscr{A}$ is a finite collection, the unions of closures and intersections of interiors
    is closed and open respectively. And closed sets are equal to its closure and open sets its interior.
    TODO: idk about (a) and (b)
\end{s}

\begin{p}{5}
    Too hard
\end{p}
\begin{s}{}
\end{s}

\begin{p}{6}
    Prove Proposition 2.30(characterization of continuity, openness, and closedness in terms of closures and interiors).\\
    Suppose $X$ and $Y$ are topological spaces, and $f: X \to Y$ is any map.\\
    (a) $f$ is continuous if and only if $f(\overline{A}) \subseteq \overline{f(A)}$ for all $A \subseteq X$.\\
    (b) $f$ is closed if and only if $f(\overline{A}) \supseteq \overline{f(A)}$ for all $A \subseteq X$.\\
    (c) f is continuous if and only if $f^{-1}(\text{Int} \, B) \subseteq \text{Int} \, f^{-1}(B)$ for all $B \subseteq Y$.\\
    (d) f is open if and only if $f^{-1}(\text{Int }B) \supseteq \text{Int }f^{-1}(B)$ for all $B \subseteq Y$.
\end{p}
\begin{s}{}
    (a) Let $B \subseteq Y$ be closed. Then $B = \overline{B}$, and also $f^{-1}(B) = f^{-1}(\overline{B})$. There exists 
    $A \subseteq X$ such that $A = f(B)$. Then $f(\overline{A}) \subseteq \overline{f(A)}$ implies 
    $\overline{A} \subseteq f(\overline{B})$. Then $f^{-1}(B) \subseteq \overline{A} \subseteq f^{-1}(\overline{B})$, and since
    $f^{-1}(B) = f^{-1}(\overline{B})$, $\overline{A} = f^{-1}(B)$ so $f^{-1}(B)$ is closed.\\
    Let $f$ be continuous. Let $f(A) \subseteq Y$, so $\overline{f(A)}$ is closed. Since $f$ is continuous,
    $f^{-1}(\overline{f(A)})$ is closed and contains $A$, thus it contains $\overline{A}$ since the closure of $A$ is the smallest
    closed set that contains $A$.\\
    (b) The proof is basically the same as part (a).\\
    (c) Let $B \subseteq Y$. Since $f$ is continuous, $f^{-1}(\text{Int }B)$ is open. And since $\text{Int }B \subseteq B$,
    $f^{-1}(\text{Int }B)$ is contained in $f^{-1}(B)$, so is contained in $\text{Int }f^{-1}(B)$.\\
    Conversely, let $B \subseteq Y$ be open. Then $B = \text{Int }B$. Then we have 
    $f^{-1}(B) \subseteq = f^{-1}(\text{Int }B) \subseteq \text{Int }f^{-1}(B)$. By definition, interiors are contained in the 
    subset, thus $\text{Int }f^{-1}(B) = f^{-1}(B)$, so the preimage is open so $f$ is continuous.\\
    (d) same as part (c)
\end{s}

\begin{p}{}
\end{p}
\begin{s}{}
\end{s}

\begin{p}{}
\end{p}
\begin{s}{}
\end{s}


\end{document}
