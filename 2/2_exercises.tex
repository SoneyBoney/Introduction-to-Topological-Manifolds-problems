%%%%%%%%%%%%%%%%%%%%%%%%%%%%%%%%%%%%%%%%%%%%%%%
%%%This is a science homework template. Modify the preamble to suit your needs. 
%The junk text is   there for you to immediately see how the headers/footers look at first 
%typesetting.


\documentclass[12pt]{article}

%AMS-TeX packages
\usepackage{amssymb,amsmath,amsthm} 
%geometry (sets margin) and other useful packages
\usepackage[margin=1.25in]{geometry}
\usepackage{graphicx,ctable,booktabs,mathrsfs}


%
%Redefining sections as problems
%
\makeatletter
\newenvironment{problem}{\@startsection
       {section}
       {1}
       {-.2em}
       {-3.5ex plus -1ex minus -.2ex}
       {2.3ex plus .2ex}
       {\pagebreak[3]%forces pagebreak when space is small; use \eject for better results
       \large\bf\noindent{P }
       }
       }
\makeatother

\makeatletter
\newenvironment{solution}{\@startsection
       {subsection}
       {2}
       {-.2em}
       {-3.5ex plus -1ex minus -.2ex}
       {2.3ex plus .2ex}
       {\pagebreak[3]%forces pagebreak when space is small; use \eject for better results
       \large\bf\noindent\emph{(sol) }
       }
       }
\makeatother

%
%Fancy-header package to modify header/page numbering 
%
\usepackage{fancyhdr}
\pagestyle{fancy}
%\addtolength{\headwidth}{\marginparsep} %these change header-rule width
%\addtolength{\headwidth}{\marginparwidth}
\chead{} 
\rhead{\thepage} 
\cfoot{} 
\renewcommand{\headrulewidth}{.3pt} 
\renewcommand{\footrulewidth}{.3pt}
\setlength\voffset{-0.25in}
\setlength\textheight{648pt}

%%%%%%%%%%%%%%%%%%%%%%%%%%%%%%%%%%%%%%%%%%%%%%%

%
%Contents of problem set
%    
\begin{document}

\title{Chapter 2 Exercises}
\author{Campinghedgehog}
\date{July 16, 2023}

\maketitle

\thispagestyle{empty}

\begin{problem}{2.4}
    (a) Suppose $M$ is a set and $d,d'$ are two different metrics on $M$. Prove that $d$ and $d'$ generate the
    same topology on $M$ if and only if the following condition is satisfied:
    for every $x \in M$ and every $r > 0$, there exist positive numbers $r_1$ and $r_2$ such that 
    $B_{r_1}^{(d')}(x) \subseteq B_{r}^{(d)}(x)$ and $B_{r_2}^{(d)}(x) \subset B_{r}^{(d')}(x)$ \\
    (b) Let $(M,d)$ be a metric space, let $c$ be a positive real number, and define a new metric $d'$
    on $M$ by $d'(x,y)=c \cdot d(x,y)$. Prove that $d$ and $d'$ generate the same topology on $M$.\\
    (c) Define a metric $d'$ on $\mathbb{R}^n$ by $d'(x,y)=max\{|x_1-y_1|,\dots,|x_n-y_n|\}$. Show that
    the Euclidian metric and $d'$ generate the same topology.\\
    (d) Let $X$ be any set, and let $d$ be the discrete metric on $X$. Show that $d$ generates the discrete
    topology.\\
    (e) Show that the discrete metric and the Euclidian metric generate the same topology on the set
    $\mathbb{Z}$ of integers.
\end{problem}
\begin{solution}{}
    (a) Assume $d,d'$ generate the same topology. Then $r_1=r_2=r$ works.\\
    Conversely assume the condition holds. We want to show that an open set with respect to $d$ is also open
    in $d'$ and vice versa. Let $U$ be open in $d$. Let $B_{r}^{d} \subseteq U$ be open. Then from 
    the hypothesis there exists an $r_1$ such that $B_{r_1}^{(d')}(x) \subseteq B_{r}^{(d)}(x)$. This applies 
    to all points in $U$, so it is a union of open balls wrt to $d'$, thus $U$ is open in $d'$.
    The converse follows similarly. \\
    (b) For every open ball with radius $r$ in $d'$ just divide by $c$ for the radius in $d$ and vice versa. \\
    (c) Exercise B.1 implies there exists a constant $c \le \sqrt(n)$ such that $d'(x,y) = c \cdot |x-y|$. 
    So part b implies the result.\\
    (d) Any subset can be formed since all distances are 1.\\
    (e) The Euclidian metric on the integers yields open balls containing just the integer itself. So for any
    set, the union of all the singletons containing the integer elements is open.
\end{solution}

\begin{problem}{2.9}
    Let $X$ be a topological space and let $A \subseteq X$ be any subset.\\
    (c) A point is in $\partial A$ if and only if every neighborhood of it contains both at point of A and a 
    point of $X \setminus A$.\\
    (d) A point is in $\overline A$ if and only if every neighborhood of it contains a point of $A$.\\
    (e)
\end{problem}
\begin{solution}{}
    (c) Let $x \in \partial A$ and $U$ be an open neighborhood of $x$. If $U \cap A = \emptyset$ then $U$ is 
    in the exterior of $A$ and so is $x$ so $x \notin \partial A$. The same goes for $X \setminus A$.\\
    Similarly if a neighborhood of $x$ is contained in either Int $A$ or Ext $A$, then the neighborhood is
    completely contained in it, which is disjoint from $\partial A$.\\
    (d) $\overline A$ is either the interior of $A$ or the boundary $A$ both of which every neighborhood 
    contains a point of $A$.\\
    ()
\end{solution}

\begin{problem}{2.18}
    Prove parts (a)-(c) of Proposition 2.17
\end{problem}
\begin{solution}{}
    (a) Singleton is closed, and the domain, i.e. the whole of $X$ is also closed. \\
    (b) trivial\\
    (c) Open subset of $X$ inherits the topology of $X$, so the restriction is also continuous.
\end{solution}



\begin{problem}{}
\end{problem}
\begin{solution}{}
\end{solution}

\end{document}
